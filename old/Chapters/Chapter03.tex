\chapter{Espressioni regolari}
Le espressioni regolari sono un formalismo comunamente utilizzato per
definire linguaggi semplici.

\begin{theorem}
  La definizione induttiva di un espressione regolare su un alfabeto $A$:
  \paragraph{Base:}
  \begin{itemize}
    \item $0$ è un espressione regolare di $A$;
    \item $\lambda$ è un espressione regolare di $A$;
    \item per ogni $\sigma\in A$, $\sigma$ è un espressione regolare
      in $A$.
  \end{itemize}
  
  \paragraph{Passo induttivo:}
  \begin{itemize}
    \item se $e_1$ ed $e_2$ sono espressioni regolare di $A$,

      allora $e_1|e_2$ è un espressione regolare di $A$;
    \item se $e_1$ ed $e_2$ sono espressioni regolare di $A$,

      allora $e_1 e_2$ è un espressione regolare di $A$;
    \item se $e$ è un espressione regolare di $A$,

      allora $e^\star$ è un espressione regolarare di $A$.
  \end{itemize}
\end{theorem}

\paragraph{Esercizio}
Scrivere un \textbf{REGEX} che esprima i nomi di variabili permessi.
\[
  L_{id} = (\{'a',\dots,'z'\}\cup\{'A',\dots,'Z'\})\cdot
    \{'a',\dots,'z'\}\cup\{'A',\dots,'Z'\}\cup\{'0',\dots,'9'\}*
\]
\[
  e_{id}=(a|\dots|z|A|\dots|Z)(a|\dots|z|A|\dots|Z|0|\dots|9)*
\]

\section{Semantica}
La semantica di un espressione regolare in $A$ è un liguaggio su $A$:
\begin{itemize}
  \item $\emptyset \rightsquigarrow$ insieme vuoto;
  \item $\epsilon \rightsquigarrow \{\epsilon\}$;
  \item $\sigma \rightsquigarrow \{"\sigma"\},\forall\sigma\in A$;
  \item $e_1\mid e_2\rightsquigarrow$ unione delle semantiche di $e_1$ ed
    $e_2$;
  \item $e_1 e_2\rightsquigarrow$ concatenzatione delle semantiche di $e_1$
    ed $e_2$.
\end{itemize}

\section{Sintassi concreta delle espressioni regolari}
\subsection{Precedenza ed associatività}
\begin{itemize}
  \item la \textbf{stella di Kleene} ha priorità sulla concatenzazione e
    l'unione;
  \item la concatenazione ha precedemza sull'unione;
  \item la concatenazione e l'unione sono associative a sinistra.
\end{itemize}

\subsection{Operatori derivati}
\begin{itemize}
  \item $e+=ee+$: (una o più volte e);
  \item $\epsilon$: è rappresentata dalla stringa vuota: $a|\epsilon$ diventa
    $a|$;
  \item $e?=|e$: ($e$ è opzionale, ovvero uno o nessuno);
  \item $[a0B]$: uno qualsiasi dei caratteri nella quadre ($a|0|B$);
  \item $[b-d]$: uno qualsiasi dei caratteri nel range tra le quadre ($b|c|d$);
  \item $[a0B]|[b-d]$: può essere scritto come $[a0Bb-d]$;
  \item $[\wedge\dots]$: qualsiasi carattere ad eccezioni di $\dots$ (esempio:
    $[^a0Bb-d]$ qualsiasi carattere ad eccezione di $a,0,B,b,c,d$).
\end{itemize}

\subsection{Caratteri speciali in JAVA}
\begin{itemize}
  \item $.$ rappresenta ogni carattere;
  \item $\backslash$ è il carattere di \emph{escape}, per dare un significato speciale
    ai caratteri regolari, oppure un significato ordinario per caratteri
    speciali.
\end{itemize}

\subsubsection{Caratteri speciali cui dare un significato ordinario}
Esempi:
\[
  |,*,+,?,.,,(,),[,],-,\wedge
\]
\paragraph{Semantiche:}
\begin{itemize}
  \item $\backslash.\rightsquigarrow\{"."\}$,
  \item $\backslash\backslash\rightsquigarrow\{"\backslash"\}$,
  \item $.\rightsquigarrow\{s\mid s\text{ ha lunghezza }1\text{, l'insieme di
    tutti i caratteri} \}$,
  \item $\backslash\rightsquigarrow$ non è sintatticamente corretto.
  \item $n\rightsquigarrow\{"n"\}$,
  \item $\backslash n\rightsquigarrow\{"linefeed"\}$,
\end{itemize}

\subsubsection{Caratteri cui dare un significato speciale}
\begin{itemize}
  \item $\backslash t$: tab;
  \item $\backslash n$; newline;
  \item $\backslash s$: qualsiasi spazio vuoto;
  \item $\backslash S$: qualsiasi spazio non vuoto;
  \item $\backslash d$: qualsiasi carattere numerico ($[0-9]$);
  \item $\backslash D$: qualsiasi carattere non numerico
    ($[\wedge0-9]$);
  \item $\backslash w$: qualsiasi parola ($[[a-zA-Z\_0-9]]$);
  \item $\backslash W$: qualsiasi carattere che non sia una parola
    ($[\wedge\backslash w]$).
\end{itemize}

\begin{theorem}
  Un linguaggio si dice \emph{regolare} se può essere definito da un'
  espressione regolare.
\end{theorem}

\section{Analisi lessicale}
\begin{theorem}
  Un \emph{lexeme} è una sottostringa considerata come un'unità sintattica.
\end{theorem}

\begin{theorem}
  \emph{L'analisi lessicale} affronta il problema della decomposizione di
  una stringa in un \emph{lexeme}.
\end{theorem}

\begin{theorem}
  Un \emph{lexer} o \emph{scanner} è un programma che esegue l'analisi
  lessicale e genera \emph{lexemes}.
\end{theorem}

\paragraph{Esempio in C:}
La stringa $"x2=042;$ è decomposta nei \emph{lexemes} seguenti:
\begin{itemize}
  \item $"x2"$;
  \item $"="$;
  \item $"042"$;
  \item $";"$.
\end{itemize}

\subsection{Token}
Un \emph{token} è una nozione di \emph{lexeme} pià astratta; ad un \emph{token}
corrisponde sempre un \emph{lexeme}.

In alcuni casi un \emph{token} può mantenere informazioni sulla sematica,
come i valori dei numeri.

Un \emph{tokenizer} è un programma che esegue l'analisi lessicale e genera
\emph{token}.

\paragraph{Esempio in C:}
La stringa $"x2=042;"$ è decomposta nei token seguenti:
\begin{itemize}
  \item \emph{IDENTIFIER}: con il nome $"x2"$;
  \item \emph{ASSIGN\_OP};
  \item \emph{INT\_NUMBER}: con il valore di $34$;
  \item \emph{STATEMENT\_TERMINATOR}.
\end{itemize}

\section{Linguaggi regolari}
\begin{theorem}
  Un \emph{linguaggio regolare} è un linguaggui definibile con un espressione
  regolare.

  I \emph{linguaggi regolari} possono essere definiti in altre maniere
  equivalenti:
  \begin{itemize}
    \item con una grammatica regolare a destra o sinistra, anche chiamata
      \emph{lineare};
    \item con una serie di \emph{automata} non deterministica o deterministica
      finita (\textbf{NFA} o \textbf{DFA}).
  \end{itemize}
\end{theorem}
\subsection{Limitazioni}
I linguaggi regolari sono linguaggi semplici. Esempio:
\begin{itemize}
  \item il linguaggio degli identificatori;
  \item il linguaggio dei numeri.
\end{itemize}

Le espressioni regolari possono definire le unità che costituiscono la
sintassi di un linguaggio di programmazione, ma non possono definire la
sintassi dell'intero linguaggio.

\subsection{Esempi di linguaggi non regolari}
Il linguaggio di espressioni con numeri naturali, addizione binaria e
moltiplicazione e parentesi \textbf{non possono} essere definiti da un
espressione regolare:
il  problema è posto dalle parentesi, per cui se venissero rimosse, allora il
linguaggio sarebbe regolare.

Un altro esempio di linguaggio semplice non regolare:
\[
  \{"a"^n"b"^n\mid n\in\N\}=\{"","ab","aabb","aaabbb",\dots\}.
\]
