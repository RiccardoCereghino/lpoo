\section{Array in Java}
Gli array in Java sono oggetti modificabili speciali, hanno una componente
chiamata \emph{indice}, la quale è una variabile di istanza senza nome.

Gli array possono essere creati solo dinamicamente, alla loro crazione la
lunghezza deve essere specificata e la stessa non può cambiare a run time,
difatti \emph{length} è di tipo \emph{final}.

I componenti sono inizializzati con valori di default e sono referenziati con
indici da $0$ a $length-1$; $T[]$ è il tipo di array con tipo di componente
$T$.

\begin{lstlisting}[
  language=Java,
  escapeinside={(£}{£)},
  caption={
    Esempio di inizializzazione di array in Java
  }
]
public class ArrayUtils {
  public static int sum(int[] a) {
    int sum = 0;
    for (int el : a)
      sum += el;
    return sum;
  }
}

int [] a = {1, 2, 3};
assert ArrayUtils.sum(a) == 6;
assert ArrayUtils.sum(new int[] {1,2,3,4}) == 10;
\end{lstlisting}

\begin{lstlisting}[
  language=Java,
  escapeinside={(£}{£)},
  caption={
    Esempio di array multidimensionali in Java
  }
]
int [][] mat1 = new int[3][2];
assert mat1.length == 3;
for (int[] row : mat1) {
  assert row.length == 2;
  for (int el : row)
    assert el == 0;
}

int[][] mat2 = new int[3][];
assert mat2.length == 3;
for (int [] row : mat2)
  assert row == null;

int[][] mat3 = { (1,1}, {1,2,1}, {1,3,3,1} };
\end{lstlisting}

\section{Main methods in Java}
L'esecuzione di un programma Java può cominciare solamente da una classe con
un metodo main, della forma:

\emph{public static void main(String[] args)\{...\}}

\section{Standard output in Java}
Lo standard output si ottiene dalla chiamata a \emph{System.out}, \emph{System}
è una classe predefinita (di tipo \emph{String}), \emph{System.out} è una
variabile di classe \textbf{final} ed è di tipo \emph{PrintStream}, la quale
è una classe predefinita della libreria Java.

\section{Modularità su larga scala}
In Java si possono utilizzare moudli o pacchetti e sottopacchetti per contenere
ed esportare logicamente classi.

\section{Packages}
Le classi pubbliche definite nei pacchetti possono essere accedute dall' 
esterno, mentre le classi private no.
Sottopacchetti possono essere contenuti nei pacchetti, in una struttura 
gerarchica e sono definite diverse \textbf{unità di compilazione}.

Un unità di compilazione è composta da $3$ parti:
\begin{itemize}
  \item dichiarazione del \textbf{package}, se non presente l'unità sarà parte 
    di un pacchetto senza nome;
  \item dichiarazioni di \textbf{import};
  \item dichiarazioni di classi di primo livello.
\end{itemize}

\begin{lstlisting}[
  language=Java,
  escapeinside={(£}{£)},
  caption={
    Esempio di un unità di compilazione in Java
  }
]
package shapes;

import java.awt.Color;

class Point {
...
}

public class ColoredLine {
  private Point a;
  private Point b;
  private Color color = Color.BLACK;
  ...
  public COlor getColor() { return this.color; }
  public void setColor ( Color color ) { this.color = color; }
)
\end{lstlisting}

\subsection{Pacchetti e sottopacchetti}
I pacchetti hanno namespaces gerarchici, dichiarazioni nello stesso pacchetto
devono avere nomi diversi, ma in pacchetti diversi possono essere uguali.
Pacchetti e sottopacchetti possono contenere solo sottopacchetti, i 
sottopacchetti devono avere un nome che rifletta il percorso di sistema.

\subsubsection{Imports}
Si possono usare \emph{type imports}, per fare riferimento ad una classe 
dichiarata in un altro pacchetto con il suo nome.
\begin{lstlisting}[
  language=Java,
  escapeinside={(£}{£)},
  caption={
    Import in java
  }
]
import java.util.Scanner;
import java.lang.*;
\end{lstlisting}

Inoltre si possono importare staticamente metodi così da abbreviarne l'utilizzo
\begin{lstlisting}[
  language=Java,
  escapeinside={(£}{£)},
  caption={
    Esempio di static import in Java
  }
]
import static java.lang.System.out;

...
out("Posso usare System.out scrivendo solo out");
\end{lstlisting}

\section{Oggetti e tipi primitivi}
Assegnare tipi agli oggetti permette che i valori possano essere gestiti
uniformamente attraverso referenze.
Le conversioni tra primitivi a valori di tipo oggetto si dicono \textbf{boxing}
mentre quelle da oggetto a primitivo si dicono \textbf{unboxing}.

\begin{lstlisting}[
  language=Java,
  escapeinside={(£}{£)},
  caption={
    Esempi di oggetti e tipi primitivi in Java
  }
]
assert 5 / 2 == 2;
assert 5 / 2. == 2.5;
Integer i = 5;
assert i * 2 == 10;
assert i * i == 25;
assert i / 2. == 2.5;
\end{lstlisting}

\section{Boxing, unboxing ed efficienza}
\begin{lstlisting}[language=Java, escapeinside={(£}{£)}, caption={
Boxing, unboxing in Java
}
]
public static int sum (Integer[] ints) {
  int s = 0;
  for (int n : ints) { s += n; }
  return s;
}

public static Integer sumInteger(Integer[] ints) {
  Integer s = 0;
  for (Integer n : ints) { s += n; }
  return s;
}

public static void main(String[] args) {
  assert sum(new Integer[] {1,2,3,4} == 10;
  sum(new int[]{1,2,3,4});
}
\end{lstlisting}

\section{Stringhe}
Le stringhe in Java sono oggetti immutabili, delimitate dal carattere $"$ e
devono essere contenute in una singola linea.

Le escape sequences sono:
\begin{itemize}
  \item $\\b$ backspace;
  \item $\\t$ horizontal tab;
  \item $\\n$ linefeed;
  \item $\\f$ form feed;
  \item $\\r$ carriage return;
  \item $\\"$ double quote;
  \item $\\'$ single quote;
  \item $\\\\$ backslash.
\end{itemize}

Le stringhe devono essere comparate con metodi di istanza,
\emph{compareTo(String anotherString)} oppure \emph{equals(Object anObject)}.

\section{Regular expressions in Java}
Le librerie che implementano le \emph{RegEx} in Java sono:
\begin{itemize}
  \item java.util.regex.Pattern
  \item java.util.regex.Matcher
  \item java.lang.String
\end{itemize}

\subsection{Pattern}
Le istanze di \emph{Pattern} sono oggetti immutabili rappresentanti RegEx,
creati da stringhe usando \emph{Pattern compile (String regex)}.

\subsection{Matcher}
I matcher sono oggetti mutabili con una sequenza di input ed un pattern,
funziona su di una sottosequenza di sequenze di input chiamate \textbf{regioni}
i cui limiti possono essere modificati con:
\emph{Matcher region (int start, int end)}
e la sequenza di input può essere cambiata con:
\emph{Matcher reset(CharSequence input)}.

\subsection{Operazioni di match}
\begin{itemize}
  \item \emph{boolean matches()}: prova ad eseguire un match su tutta la regione
    utilizzando il pattern;
\item \emph{boolean lookinhAt()}: prova ad eseguire un match su una
  sottosequenza della regione all'inizio;
\item \emph{boolean find()}: prova a trovare la sottosequenza successiva della
  sequenza di input che matcha il pattern.
\end{itemize}

\begin{lstlisting}[language=Java, escapeinside={(£}{£)}, caption={
RegEx in Java
}
]
Pattern pt = Pattern.compile("[A-Z][a-z]+");
Matcher mt = pt.matcher("Java");
assert mt.matches();
mt.reset("Java language");
assert !mt.matches();
assert mt.lookingAt();
mt.reset("language Java");
assert !mt.matches();
assert !mt.lookingAt();
assert mt.find();
\end{lstlisting}

\section{Operazioni di query}
\begin{itemize}
  \item \emph{int start()}: ritorna l'indice d'inizio del match precedente;
  \item \emph{int end()}: ritorna l'indice dopo l'ultimo carattere matchato;
  \item \emph{String group()}: ritorna la stringa matchata dal match
    precedente.
\end{itemize}

\begin{lstlisting}[language=Java, escapeinside={(£}{£)}, caption={
Esempi di query RegEx in Java
}
]
Pattern pt = Pattern.compile("[A-z][a-z]+");
Matcher mt = pt.matcher("Java Language");
assert !mt.matches();
assert mt.lookingAt();
assert mt.start() == 0;
assert mt.end() == 4;
assert mt.group.equals("Java");
mt.region(mt.end(),mt.regionEnd());
assert !mt.matches();
assert !mt.lookingAt();
assert mt.find();
assert mt.start() == 5;
assert mt.end() == 13;
mt.group().equals("Language");
\end{lstlisting}

Se una query incontra error lancerà l'errore \emph{IllegalStateException}.

\section{MatchResult}
Il risultato dell'ultima operazione può essere ritornato con il metodo di
istanza \emph{MatchResult toMatchResult()}, il risultato non interagisce con le
operazioni che si possono effettuare sul matcher.

\begin{lstlisting}[language=Java, escapeinside={(£}{£)}, caption={
MatchResult in Java
}
]
Pattern pt = Pattern.compile("[A-Z][a-z]+");
Matcher mt = pt.matcher("Java Language");
assert mt.lookingAt();
MatchResult res = mt.toMatchResult();
mt.region(res.end(), mt.regionEnd());
assert res.start() == 0;
assert res.end() == 4;
assert res.group().equals("Java");
assert mt.start() == 0;
\end{lstlisting}


\section{Capturing groups}
Le parentesi forzano la precedenza, oppure possono essere usate per i capturing
groups, i quali sono indiciati da sinistra a destra cominciando da $1$, il
gruppo $0$ è riservato per l'intero pattern.

\begin{lstlisting}[language=Java, escapeinside={(£}{£)}, caption={
Esempio di capturing groups in Java
}
]
Pattern pt = Pattern.compile("(0|[1-9]\\d*)([Ll]?)");
Matcher mt = pt.matcher("42L");
mt.lookingAt();
MatchResult res = mt.toMatchResult();
assert res.group(0).equals("42L");
assert res.group(1).equals("42");
assert res.group(2).equals("L");
mt.reset("42");
mt.lookingAt();
res = mt.toMatchResult();
assert res.group(0).equals("42");
assert res.group(1).equals("42");
assert res.group(2).equals("");
\end{lstlisting}

\section{Supporto di Java per le RegEx}
\begin{itemize}
  \item Operatori logici:
    \begin{itemize}
      \item $XY$: $X$ seguito da $Y$ (concatenazione);
      \item $X|Y$: $X$ oppure $Y$ (unione);
      \item $(X)$: X, un capturing group, le parentesi forzano la precedenza;
    \end{itemize}
  \item operatori postfissi:
    \begin{itemize}
      \item $X?$: opzionale;
      \item $X*$: $X$ $0$ o più volte (stella di Kleene);
      \item $X+$: $X$ $1$ o più volte;
    \end{itemize}
  \item caratteri:
    \begin{itemize}
      \item $x$: il carattere $x$ (se non è un carattere speciale);
      \item $\\$: niente, ma funziona come carattere di escape;
      \item $\\t$: il carattere tab;
      \item $\\n$: il carattere newline;
      \item $\\r$: il carattere carriage-return;
    \end{itemize}
  \item classi di caratteri:
    \begin{itemize}
      \item $[abc]$: $a,b,$ o $c$ (classe semplice);
      \item $[^abc]$: qualsiasi carattere eccetto $a,b,c$;
      \item $[a-zA-Z]$: range di caratteri;
    \end{itemize}
  \item classi di caratteri predefinite:
    \begin{itemize}
      \item $.$: qualsiasi carattere (eccetto i terminatori di linea, a meno
        che il flag \emph{DOTALL} sia specificato);
      \item $\\s$: il carattere whitespace;
    \end{itemize}
  \item matcher di boundaries:
    \begin{itemize}
      \item $\wedge$: l'inizio di una linea;
      \item $\$$: la fine di una linea;
    \end{itemize}
  \item capturing con nome e senza:
    \begin{itemize}
      \item $(?<name>X)$: definisce il gruppo di cattura $X$;
      \item $(?:X)$: un gruppo di non cattura $X$.
    \end{itemize}
\end{itemize}

\begin{lstlisting}[language=Java, escapeinside={(£}{£)}, caption={
Supporto in Java per le RegEx
}
]
Pattern pt = Pattern.compile("(\\d\\d\\d)?");
Matcher mt = pt.matcher("234");
assert mt.lookingAt();
assert mt.end() == 3;

pt = Pattern.compile("\\d+");
Matcher mt = pt.matcher("234");
assert mt.lookingAt();
assert mt.end() == 3;

pt = Pattern.compile("\\d*");
Matcher mt = pt.matcher("234");
assert mt.lookingAt();
assert mt.end() == 3;

pt = Pattern.compile("(\\d\\d)?(\\d\\d\\d)?");
Matcher mt = pt.matcher("234");
assert mt.lookingAt();
assert mt.end() == 2;

pt = Pattern.compile("\\d\\d|\\d\\d\\d?");
Matcher mt = pt.matcher("234");
assert mt.lookingAt();
assert mt.end() == 2;
\end{lstlisting}

