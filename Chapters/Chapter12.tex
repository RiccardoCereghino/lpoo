\section{Array in Java}
Gli array in Java sono oggetti modificabili speciali, hanno una componente
chiamata \emph{indice}, la quale è una variabile di istanza senza nome.

Gli array possono essere creati solo dinamicamente, alla loro crazione la
lunghezza deve essere specificata e la stessa non può cambiare a run time,
difatti \emph{length} è di tipo \emph{final}.

I componenti sono inizializzati con valori di default e sono referenziati con
indici da $0$ a $length-1$; $T[]$ è il tipo di array con tipo di componente
$T$.

\begin{lstlisting}[
  language=Java,
  escapeinside={(£}{£)},
  caption={
    Esempio di inizializzazione di array in Java
  }
]
public class ArrayUtils {
  public static int sum(int[] a) {
    int sum = 0;
    for (int el : a)
      sum += el;
    return sum;
  }
}

int [] a = {1, 2, 3};
assert ArrayUtils.sum(a) == 6;
assert ArrayUtils.sum(new int[] {1,2,3,4}) == 10;
\end{lstlisting}

\begin{lstlisting}[
  language=Java,
  escapeinside={(£}{£)},
  caption={
    Esempio di array multidimensionali in Java
  }
]
int [][] mat1 = new int[3][2];
assert mat1.length == 3;
for (int[] row : mat1) {
  assert row.length == 2;
  for (int el : row)
    assert el == 0;
}

int[][] mat2 = new int[3][];
assert mat2.length == 3;
for (int [] row : mat2)
  assert row == null;

int[][] mat3 = { (1,1}, {1,2,1}, {1,3,3,1} };
\end{lstlisting}

\section{Main methods in Java}
L'esecuzione di un programma Java può cominciare solamente da una classe con
un metodo main, della forma:

\emph{public static void main(String[] args)\{...\}}

\section{Standard output in Java}
Lo standard output si ottiene dalla chiamata a \emph{System.out}, \emph{System}
è una classe predefinita (di tipo \emph{String}), \emph{System.out} è una
variabile di classe \textbf{final} ed è di tipo \emph{PrintStream}, la quale
è una classe predefinita della libreria Java.

\section{Modularità su larga scala}
In Java si possono utilizzare moudli o pacchetti e sottopacchetti per contenere
ed esportare logicamente classi.

\section{Packages}
Le classi pubbliche definite nei pacchetti possono essere accedute dall' 
esterno, mentre le classi private no.
Sottopacchetti possono essere contenuti nei pacchetti, in una struttura 
gerarchica e sono definite diverse \textbf{unità di compilazione}.

Un unità di compilazione è composta da $3$ parti:
\begin{itemize}
  \item dichiarazione del \textbf{package}, se non presente l'unità sarà parte 
    di un pacchetto senza nome;
  \item dichiarazioni di \textbf{import};
  \item dichiarazioni di classi di primo livello.
\end{itemize}

\begin{lstlisting}[
  language=Java,
  escapeinside={(£}{£)},
  caption={
    Esempio di un unità di compilazione in Java
  }
]
package shapes;

import java.awt.Color;

class Point {
...
}

public class ColoredLine {
  private Point a;
  private Point b;
  private Color color = Color.BLACK;
  ...
  public COlor getColor() { return this.color; }
  public void setColor ( Color color ) { this.color = color; }
)
\end{lstlisting}

\subsection{Pacchetti e sottopacchetti}
I pacchetti hanno namespaces gerarchici, dichiarazioni nello stesso pacchetto
devono avere nomi diversi, ma in pacchetti diversi possono essere uguali.
Pacchetti e sottopacchetti possono contenere solo sottopacchetti, i 
sottopacchetti devono avere un nome che rifletta il percorso di sistema.

\subsubsection{Imports}
Si possono usare \emph{type imports}, per fare riferimento ad una classe 
dichiarata in un altro pacchetto con il suo nome.
\begin{lstlisting}[
  language=Java,
  escapeinside={(£}{£)},
  caption={
    Import in java
  }
]
import java.util.Scanner;
import java.lang.*;
\end{lstlisting}

Inoltre si possono importare staticamente metodi così da abbreviarne l'utilizzo
\begin{lstlisting}[
  language=Java,
  escapeinside={(£}{£)},
  caption={
    Esempio di static import in Java
  }
]
import static java.lang.System.out;

...
out("Posso usare System.out scrivendo solo out");
\end{lstlisting}

\section{Oggetti e tipi primitivi}
Assegnare tipi agli oggetti permette che i valori possano essere gestiti
uniformamente attraverso referenze.
Le conversioni tra primitivi a valori di tipo oggetto si dicono \textbf{boxing}
mentre quelle da oggetto a primitivo si dicono \textbf{unboxing}.

\begin{lstlisting}[
  language=Java,
  escapeinside={(£}{£)},
  caption={
    Esempi di oggetti e tipi primitivi in Java
  }
]
assert 5 / 2 == 2;
assert 5 / 2. == 2.5;
Integer i = 5;
assert i * 2 == 10;
assert i * i == 25;
assert i / 2. == 2.5;
\end{lstlisting}
