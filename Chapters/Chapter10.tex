\chapter{OCaml}
OCaml deriva da \textbf{ML}, è un linguaggio multi-paradigma con un approccio
puramente funzionale, è staticamente tipato con inferenza di tipo:
\begin{itemize}
  \item gli errori di tipo sono rilevati staticamente;
  \item i tipi possono essere omessi nei programmi.
\end{itemize}

\begin{lstlisting}[language=Caml, caption={Sintassi}]
Exp ::= ID | NUM | Exp Exp | 'fun' Pat+ '->' Exp | UOP Exp | Exp BOP Exp | '(' Exp ')'
Pat ::= ID // pattern semplificato
\end{lstlisting}

\paragraph{Commenti:}
\begin{itemize}
  \item \textbf{ID} rappresenta gli identificatori di variabile $[a-zA-Z\_]
    [\backslash w']*$;
  \item \textbf{NUM} rappresenta i numeri naturali:
    \[
      0[bB][01][01\_]*|0[oO][0-7][0-7\_]*|0[xX][0-9a-fA-F][0-9a-fA-F\_]*|\backslash d[\backslash d\_]*
    \];
  \item \textbf{UOP} rappresenta operatori aritmetici unari $[+-]$;
  \item \textbf{BOP} rappresenta operatori aritmetici binari $[+-*\backslash]|mod$;
  \item \textbf{Pat} rappresenta i patterns, per semplicità si possono
    associare agli identificatori.
\end{itemize}

\section{Funzioni ed applicazioni}
\begin{itemize}
  \item esempi di funzioni anonime:
    \begin{lstlisting}[language=Caml, caption={Esempio di funzioni anonime}]
    fun x -> x+1 (* funzione di incremento *)
    fun x y -> x+y (* funzione di addizione *)
    \end{lstlisting}
  \item applicazione di funzioni:
    \begin{lstlisting}[language=Caml, caption={Esempio di applicazione di funzioni}]
    (fun x -> x+1) 3 (* il risultato sara 4 *)
    \end{lstlisting}
\end{itemize}

Inoltre prendendo in esempio:
\begin{lstlisting}[language=Caml, caption={Esempio di applicazioni}]
  exp1 exp2
\end{lstlisting}
\begin{itemize}
  \item l'evaluazione di \textbf{exp1} si aspetta come valore di ritorno una
    funzione $f$;
  \item l'evaluazione di \textbf{exp2} si aspetta come valore di ritorno un
    argomento valido $a$;
  \item l'evaluazione di \textbf{exp1 exp2} ritorna $f(a)$ ($f$ applicato ad
    $a$).
\end{itemize}

\section{Regole di precedenza ed associatività}
\begin{itemize}
  \item Si usano le regole standard per le espressioni aritmetiche;
  \item l'applicazione è associativa a sinistra:
    \begin{lstlisting}[language=Caml, caption={Esempio di associatività}]
    (fun x y -> x+y) 3 4 (* equivale a ((fun x y -> x+y) 3) 4 *)
    \end{lstlisting}
  \item l'applicazione ha precedenza maggiore rispetto gli operatori binari:
    \begin{lstlisting}[language=Caml, caption={Esempio di precedenza}]
    (fun x -> x*2) 1+2 (* equivale a ((fun x -> x*2) 1) +2 *)
    1+(fun x -> x*2) 1+2 (* equivale a 1+((fun x -> x*2) 1) +2 *)
    \end{lstlisting}
  \item le funzioni anonime hanno piorità più bassa rispetto le applicazioni
    e gli operatori binari:
    \begin{lstlisting}[language=Caml, caption={Esempio di precedenza}]
    fun x -> x*2 (* equivale a fun x -> x*2) *)
    fun f a -> f a (* equivale a fun f a -> (f a) *)
    \end{lstlisting}
  \item i casi limite:
    \begin{lstlisting}[language=Caml, caption={Casi limite di precedenza}]
    f + 3 (* addizione *)
    f (+3) (* applicazione *)
    f - 3 (* sottrazione *)
    f (-3) (* applicazione *)
    + f 3 (* equivale a +(f 3) *)
    - f 3 (* equivale a -(f 3) *)
    \end{lstlisting}
\end{itemize}

\section{Una sessione REPL (Read Eval Print Loop)}
\begin{lstlisting}[language=Caml, caption={I tipi possono essere inferiti dall'interprete}]
# 42;;
- : int = 42
# fun x->x*2;;
- : int -> int = <fun>
# (fun x->x+1) 2;;
- : int = 3
\end{lstlisting}

\section{Sintassi}
\begin{lstlisting}[language=Caml, caption={Grammatica BNF}]
Type ::= 'int' | Type '->' Type
\end{lstlisting}

\subsection{Terminologia}
\begin{itemize}
  \item \textbf{int} è il tipo primitivo degli interi;
  \item \textbf{int} -> \textbf{int} è un tipo composito;
  \item \textbf{->} è un tipo \textit{costruttore}, usato per costruire tipi
    compositi da tipi più semplici;
  \item i tipi costruiti con il costruttore freccia (->) sono chiamati \emph{
    arrow types} o \emph{tipi di funzione}.
\end{itemize}

\paragraph{Significato del tipo freccia}
$t_1 -> t_2$ identifica il tipo di funzioni da $t_1$ a $t_2$ che:
\begin{itemize}
  \item piò essere applicazo ad un singolo argomento di tipo $t_1$;
  \item ritorna sempre un valore di tipo $t_2$.
\end{itemize}

\subparagraph{Osservazioni:}
\begin{itemize}
  \item il costruttore di tipo freccia è associativo a destra.
    \begin{lstlisting}[language=Caml, caption={Associatività a destra dell'operatore freccia}]
    int->int->int = int->(int->int)
    \end{lstlisting}
  \item un tipo costruttore costruisce sempre un tipo diverso rispetto a quello
    dei propri componenti:
    \[
      t_1 \text{->} t_2 \neq t_1 \qquad t_1 \text{->} t_2 \neq t_2
    \]
  \item due tipi freccia sono uguali se sono costruiti dallo stesso tipo di
    componenti:
    \[
      t_1 \text{->} t_2 = t_3 \text{->} t_4 \text{ se e solo se } t_1 = t_3 \cap t_2 = t_4
    \]
  \item dai punti prima deduciamo:
    \begin{lstlisting}[language=Caml, escapeinside={(*}{*)}]
    int->int->int (*$\neq$*) int->(int->int)
    \end{lstlisting}
\end{itemize}

\subsection{Funzioni di alto ordine}
\[
  \textbf{fun } pat_1 pat_2 \dots pat_n \text{->} exp
\]
è un abbreviazione di:
\[
  \textbf{fun } pat_1 \text{->}\textbf{fun }pat_2 \text{->}\dots \textbf{fun }pat_n \text{->} exp
\]
\section{Tuple}
\begin{lstlisting}[language=Caml, escapeinside={(*}{*)}, caption={Nuove
produzioni per Exp e Pat}]
Exp ::= '(' ')' | Exp ',' Exp
Pat ::= '(' ')' | '(' Pat (',' Pat)* ')'
\end{lstlisting}

\begin{lstlisting}[language=Caml, escapeinside={(*}{*)}, caption={Nuova
produzione per Type}]
Type ::= 'unit' | Type '*' Type
\end{lstlisting}

\subsection{Precedenza ed associatività}
\begin{itemize}
  \item l'operatore \emph{tuple} ha precedenza più bassa rispetto gli altri
    operatori;
  \item l'operatore \emph{tuple} non è associativo nè a destra nè a sinistra;
  \item il costruttore $*$ ha precedenza maggiore del costruttore $->$;
  \item il costruttore $*$ non è associativo nè a destra nè a sinistra.
\end{itemize}

\begin{lstlisting}[language=Caml, escapeinside={(*}{*)}, caption={Esempi
di tuple}]
# ()
- : unit = ()

# 1,2,3
- : int * int * int = (1,2,3)

# (1,2),3
- : (int * int) * int = ((1,2),3)

# 1,(2,3)
- : int * (int * int) = (1,(2,3))

# fun() -> 3
- : unit -> int = <fun>

# fun ((x,y),z)->x*y*z
- : (int * int) * int -> int = <fun>

#fun (x. (y,z))->x*y*z
- : int * (int * int) -> int 0 <fun>
\end{lstlisting}

\section{Funzioni curry}
\begin{theorem}
  Una funzione curry (da \emph{Haskell Curry}), è una funzione di alto
  ordine con un singolo argomento che ritorna una catena di funzioni con
  un singolo argomento.

  Una funzione non curry è una funzione con argomenti multipli.
\end{theorem}

Una funzione curry può essere trasformata in una funzione non curry e
viceversa.

\begin{lstlisting}[language=Caml, escapeinside={(£}{£)}, caption={Esempi
di funzioni curry}]
(* addizione di due interi *)
fun x y -> x+y;; (* versione curry int->int->int *)
fun (x y) -> x+y;; (* versione non curry int->int->int *)
(* moltiplicazione di tre interi *)
fun x y z -> x*y*z;; (* versione curry int->int->int->int *)
fun (x y z) -> x*y*z;; (* versione non curry int->int->int->int *)
\end{lstlisting}

\subsection{Applicazione parziale}
Le funzioni curry permettono l'applicazione parziale, ovvero gli argomenti
possono essere passati uno alla volta.

Le funzioni non curry non permettono un applicazione parziale, tutti gli
argomenti devono essere passai.
\begin{lstlisting}[language=Caml, escapeinside={(£}{£)}, caption={Esempi
di applicazione parziale di funzioni curry}]
let curried_add x y=x+y;;
let uncurried_add(x,y)=x+y;;
(* computa 1+2 con la versione non curry *)
uncarried_add(1,2);;
(* computa 1+2 con l'applicazione parziale *)
let inc=curried_add 1;; (* passa l'argomento 1 e salva il risultato *)
inc 2;; (* passa l'argomento 2 e computa il risultato finale *)
\end{lstlisting}

L'applicazione parziale permette la specializzazione di funzioni: da una
funzione generica è possibile generarne di più specifiche senza
duplicazione di codice, quindi il riutilizzo e la mantenibilità sono
favoriti.

\section{Valori booleani}
Per $BOOL=false|true$.
\begin{lstlisting}[language=Caml, escapeinside={(£}{£)}, caption={Sintassi}]
Exp ::= BOOL | 'not' Exp | Exp '&&' Exp | Exp '||' Exp
Type ::= 'bool'
\end{lstlisting}

\subsection{Regole sintattiche standard}
\begin{itemize}
  \item $\&\&$ e $||$ sono associativi a sinistra;
  \item \textbf{not} ha precedenza maggiore di $\&\&$;
  \item $\&\&$ ha precedenza maggiore di $||$.
\end{itemize}

\subsection{Semantica statica}
\begin{itemize}
  \item \textbf{false} e \textbf{true} sono di tipo \emph{bool};
  \item \textbf{not} $e$ è di tipo \emph{bool} se e solo se $e$ è di tipo
    \emph{bool};
  \item il tipo di \textbf{not} $e$ non è corretto se $e$ non è di tipo
    \emph{bool} oppure se il tipo di $e$ non è corretto;
  \item $e_1\&\&e_2$ e $e_1||e_2$ sono di tipo \emph{bool} se e solo se $e_1$
    ed $e_2$ sono di tipo \emph{bool};
  \item il tipo di \textbf{not} $e_1\&\&e_2$ e $e_1||e_2$ non è corretto se
    $e_1$ o $e_2$ non sono di tipo \emph{bool} oppure se il tipo di $e_1$ o
    $e_2$ non è corretto.
\end{itemize}


Inoltre:
\begin{itemize}
  \item gli operatori $\&\&$ e $||$ sono risolti sinistra a destra;
  \item se $e_1$ ritorna \emph{false} allora $e_1\&\&e_2$ ritorna
    \emph{false}, altrimenti ritorna il valore di $e_2$;
  \item se $e_1$ ritorna \emph{true} allora $e_1||e_2$ ritorna \emph{true},
    altrimenti ritorna il valore di $e_2$.
\end{itemize}

\subsection{Espressioni condizionali}
le operazioni condizionali hanno precedenza minore di tutte le altre
operazioni.

\begin{lstlisting}[
  language=Caml,
  escapeinside={(£}{£)},
  caption={
    Espressioni condizionali
  }]
Exp ::= 'if' Exp 'then Exp 'else' Exp
\end{lstlisting}

\section{Variabili globali}
\begin{lstlisting}[
  language=Caml,
  escapeinside={(£}{£)},
  caption={
    Grammatica delle variabili globali
  }
]
Dec ::= 'let' Def (' and' Def)*
    | 'let' 'rec' FunDef (' and' FunDef)*
Def ::= Pat '=' Exp | FunDef
FunDef ::= ID Pat* '=' Exp
\end{lstlisting}

\paragraph{Esempio di variabili globali e funzioni curry}
Consideriamo i seguenti due esempi.
\begin{lstlisting}[
  language=Caml,
  escapeinside={(£}{£)},
  caption={
    Addizione di quadrati
  }
]
let rec sumsquare n = (* sumsquare si usa anche con associativo a destra*)
  if n<=0 then 0 else n*n+sumsquare(n-1);;
\end{lstlisting}

\begin{lstlisting}[
  language=Caml,
  escapeinside={(£}{£)},
  caption={
    Addizione di cubi
  }
]
let rec sumcube n = (* sumcube si usa anche con associativo a destra*)
  if n<=0 then 0 else n*n+sumcube(n-1);;
\end{lstlisting}


Si nota che sono quasi identici, quindi si può usare una funzione curry.

\begin{lstlisting}[
  language=Caml,
  escapeinside={(£}{£)},
  caption={
    Soluzione con funzione curry
  }
]
let rec gen_sum f n = (* (int -> int) -> int -> int *)
if n<=0 then 0 else f n+gen_su, f (n-1);;

let der_sumsquare = gen_sum (fun x->x*x);; (* int -> int *)
let der_sumcube = gen_sum (fun x->x*x*x);; (* int -> int *)
\end{lstlisting}

Notiamo che $gen\_sum$ può essere specializzato dato che è funzione curry
ed il primo argomento è $f$ piuttosto che $n$.

\section{Dichiarazione di variabili locali}
\begin{lstlisting}[
  language=Caml,
  escapeinside={(£}{£)},
  caption={
    Sintassi delle variabili locali
  }
]
Dec ::= 'let' Def (' and' Def)* 'in' Exp
    | 'let' 'rec' FunDef (' and' FunDef)* 'in' Exp
Def ::= Pat '=' Exp | FunDef
FunDef ::= ID Pat* '=' Exp
\end{lstlisting}

\begin{lstlisting}[
  language=Caml,
  escapeinside={(£}{£)},
  caption={
    Esempio di variabili locali
  }
]
let f x=x+1 and v=41 in f v;; (* f e v possono essere usati solo qui *)
- : int = 42
let x=1 in let x=x*2 in x*x (* dichiarazioni annidate *)
- : int = 4
\end{lstlisting}

Da notare che le dichiarazioni annidate sovrascrivono le dichiarazioni con
lo stessi $ID$.

\subsection{Scopo delle dichiarazioni statiche}
\begin{lstlisting}[
  language=Caml,
  escapeinside={(£}{£)},
  caption={
    Esempio di dicharazione statica
  }
]
let v=40;;

let f x = x*v;; (* v riferisce alla dichiarazione precedente *)

f 3;; (* ritorna 120 *)

let v=4;; (* dichiarazione di v sovrascritta *)

f 3;; (* ritorna 120 *)
\end{lstlisting}

\begin{lstlisting}[
  language=Caml,
  escapeinside={(£}{£)},
  caption={
    Miglioramento dell'esempio della somma di quadrati e cubi
  }
]
let gen_sum f = (* (int ->) -> int -> int *)
    let rec aux n = if n<=0 then 0 else f n+aux (n-1) (* int -> int )
    in aux;;
\end{lstlisting}

Non dobbiamo passare l'argomento $f$ alla funzione ricorsiva $aux$.

\clearpage
\section{Liste}
\begin{lstlisting}[
    language=Caml,
    escapeinside={(£}{£)},
    caption={
      Sintassi delle liste
    }
  ]
  Exp ::= '[' ']' | Exp '::'Exp
\end{lstlisting}

\begin{itemize}
  \item la lista vuota è rappresentata da $[]$;
  \item $hd::ts$ è la lista con la testa ($hd$) e la coda ($tl$);
  \item $[]\neq t_1::t_2$ e $t_1\neq t_1::t_2$ e $t_2\neq t_1::t_2$;
  \item $t_1::t_2=t_1^\prime::t_2^\prime$ se e solo se $t_1=t_1^\prime$ e $t_2
    =t_2^\prime$;
  \item $[e_1;e_2;\dots;e_n]$ è l'abbreviazione per $e_1::e_2::\dots::e_n$.
\end{itemize}

\subsection{Regole sintattiche}
\begin{itemize}
  \item Associatività a destra;
  \item minore precedenza degli operatori unari e binari con notazione infissa;
  \item maggiore precedenza de:
    \begin{itemize}
      \item il costruttore di tupla;
      \item espressioni di funzioni anonime ($\textbf{fun }\dots->\dots$);
      \item espressioni condizionali ($ \textbf{if }\dots \textbf{ then }\dots
        \textbf{ else }\dots$);
    \end{itemize}
  \item si può usare la notazione $[e_1;e_2;\dots e_n]$ come abbreviazione.
\end{itemize}
\paragraph{Attenzione}
L'operatore $;$ all'interno delle quadre ha le proprie regole di precedenza,
per esempio l'operatore di tupla ha maggiore precedenza.

\subsection{Tipi di costruttori per le liste}
Le liste devono essere omogenee, tutti gli elementi devono essere dello stesso
tipo.


Il costruttore unario postfisso è $list$ ha precedenza maggiore dei
costruttori $->$ e $*$.

\begin{lstlisting}[
  language=Caml,
  escapeinside={(£}{£)},
  caption={
    Esempi di liste
  }
]
[1;2] (* una lista di interi *)
- : int list = [1;2]

[true;false;true] (* una lista di booleani *)
- : bool list = [true;false;true]

[1,true] (* una lista di coppie int*bool *)
- : (int * bool) list = [(1, true)]

[[1;2];[0;3;3];[]] (* una lista di liste di interi *)
- : int list list = [[1;2];[0;3;3];[]]
\end{lstlisting}

\subsubsection{Semantica statica}
Nella sintassi concreta di \emph{OCaml}, $[]$ è di tipo $\alpha\text{ list}$ o
$' \text{a } list$.

$e_1::e_2$ ha tipo $\text{t }list$ se e solo se $e_1$ ha tipo $t$ ed $e_2$ ha
tipo $\text{t }list$.

$e_1::e_2$ non è di tipo corretto se non esiste tipo $t$.

\subsection{Concatenazione}
L'operatore binario infisso è associativo a sinistra ed ha minore precedenza
del costruttore $::$; la concatenazione \textbf{non} è un costruttore.
\begin{lstlisting}[
  language=Caml,
  escapeinside={(£}{£)},
  caption={
    Operatore binario infisso di lista
  }
]
Exp ::= Exp '@' Exp
\end{lstlisting}

\subsubsection{Semantica statica}
Una lista concatenata $e_1@e_2$ sarà di tipo lista se e solo se entrambi gli
$e$ sono liste, altrimenti non sarà di tipo corretto.

\subsection{Esempi}
\begin{lstlisting}[
  language=Caml,
  escapeinside={(£}{£)},
  caption={
    Esempi di liste
  }
]
[1;2]@[3]@[4;5;6];;
- : int list 0 [1;2;3;4;5;6]

[[1]]@[2]::[[3]];;
- : int list list = [[1]; [2]; [3]]

([1]@[2])::[[3]];;
- : int list list = [[1; 2]; [3]] 

(@)
- : 'a list -> -> 'a list = <fun>

(@) [1;2]
- : int list -> int list = <fun>

(@) [1;2] [3;4;5]
- : int list = [1; 2; 3; 4; 5]


\end{lstlisting}

\section{Pattern matching}
Le funzioni che non possono essere definite con un singolo pattern sono:
\begin{itemize}
  \item la lunghezza di una lista;
  \item la somma di tutti gli elementi di una lista;
  \item la lista con i primi due elementi scambiati.
\end{itemize}

\begin{lstlisting}[
  language=Caml,
  escapeinside={(£}{£)},
  caption={
    Nuove produzioni per Pat
  }
]
Pat ::= '[' ']' | Pat '::' Pat | '[' Pat (';' Pat)* ']'
\end{lstlisting}

\paragraph{Osservazione}
Tutte le variabili in un pattern devono essere distinte (questo
rende il controllo sui pattern più efficiente).
Inoltre i \emph{patterns} sono costruiti con costruttori, non con
altri operatori:
$x::y$ è un pattern valido, $x@y$ o $x+y$ non lo sono; i costruttori
garantiscono un unica decomposizione dei valori.

\subsection{Esempi di pattern matching}
\begin{lstlisting}[
  language=Caml,
  escapeinside={(£}{£)},
  caption={
    Primo esempio di pattern matching
  }
]
let add (x,y) = x+y;;
add (r,5);;
\end{lstlisting}
\begin{itemize}
  \item $(3,5)$ combacia con il pattern $(x,y)$ se e solo se $x=3\cap
    y=5$;
  \item se sostituiamo $x,y$ in $(x,y)$ con $3$ e $5$,
    rispettivamente, allora otteniamo il valore $(3,5)$.
\end{itemize}

\begin{lstlisting}[
  language=Caml,
  escapeinside={(£}{£)},
  caption={
    Secondo esempio di pattern matching
  }
]
let hd (h::t) = h;; (* ritorna la testa della lista *)
hd [3;5];;
\end{lstlisting}
\begin{itemize}
  \item $[3;5]$ combacia con il pattern $(h::t)$ con $3$ e $[5]$
    rispettivamente, quindi otteniamo il valore $(3::[5])=[3;5]$.
\end{itemize}

\begin{lstlisting}[
  language=Caml,
  escapeinside={(£}{£)},
  caption={
    Terzo esempio di pattern matching
  }
]
let hd (h::t) = h;;

hd[];;
\end{lstlisting}
\begin{itemize}
  \item $[]$ non combacia con $(h::t)$ per qualunque valore associato a $h$ e
    $t$;
  \item quindi $[]\neq(h::t)$ per tutti i possibili valori associati con $h$ e
    $t$;
  \item il comportamento è corretto, dato che la testa di una lista non è
    definita per la lista vuota.
\end{itemize}

\subsection{Matching di patterns multipli}
\begin{lstlisting}[
  language=Caml,
  escapeinside={(£}{£)},
  caption={
    Espressione per eseguire un match con multipli patterns
  }
]
Exp ::= 'match' Exo 'with' Pat '->' Exp ('|' Pat '->' Exp)*
\end{lstlisting}

\subsubsection{Esempi}
\begin{lstlisting}[
  language=Caml,
  escapeinside={(£}{£)},
  caption={
    Esempio di match multipli
  }
]
let rec length 1 = match 1 with
    [] -> 0
  | hd::tl -> 1+length t1;;

let rec sum 1 = match 1 with
    [] -> 0
  | hd::tl -> hd+sum tl;;

let swap 1 = match 1 with
    [] -> []
  | [x] -> [x]
  | x::y::1 -> y::x::1;;
\end{lstlisting}

\subsubsection{Sintassi}
\begin{lstlisting}[
  language=Caml,
  escapeinside={(£}{£)},
  caption={
    Sintassi di matching di patterns multipli
  }
]
match e with (£$p_1$£) -> (£$e_1$£) | (£$\dots$£) (£$p_n$£) -> (£$e_n$£)
\end{lstlisting}

\paragraph{Semantica statica}
L'espressione $e$ e tutti i patterns $p_1\dots p_n$ devono essere dello stesso
tipo, così come tutte le espressioni $e_1\dots e_n$.


Viene riportato un warning se i patterns non sono esaustivi, per esempio
se sono mancanti, oppure se un pattern non è utilizzato.

\paragraph{Semantica dinamica}
In ordine vengono calcolati:
\begin{enumerate}
  \item $e$;
  \item tutti i pattern $p_1\dots p_n$ testati da sinistra a destra, dalla cima
    in fondo;
  \item al primo \emph{match} con $p_i$, l'espressione $e_i$ è calcolata,
    con le variabili definite dal match con $p_i$;
  \item se non viene trovato un match, allora l'errore \emph{Match\_failure}
    è sollevato.
\end{enumerate}

\subsection{Decomposizione unica}
I costruttori assicurano che se esiste un march per $p$, allora esiste un unica sostituzione per le variabili in $p$.

\subsection{Costruttori per i tipi primitivi}
Tutti i litterali (tokens che rappresentano valori) sono costruttori costanti.

\subsection{Notazione abbreviata}
\begin{itemize}
  \item la carattere \textit{wildcard} $\_$ è il pattern che matcha tutti i
    valori quando nessuna variabile è necessaria;
  \item $\textbf{function } p_1->e_1|\dots|p_n->e_n$ abbrevia la notazione:
    \begin{lstlisting}[
      language=Caml,
      escapeinside={(£}{£)},
    ]
    fun var -> match var with (£$p_1$£) -> (£$e_1$£) | (£$\dots$£) | (£$p_n$£)(£$e_n$£)
    \end{lstlisting}
  \item $p \textbf{ as } id$: un pattern (o sotto-pattern) $p$ può essere
    associato con un $id$ per fare riferimento al valore trovato direttamente.
\end{itemize}

\subsection{Esempi di pattern matching funzionanti}
\begin{lstlisting}[
  language=Caml,
  escapeinside={(£}{£)},
  caption={
    Esempi di pattern matching funzionanti
  }
]
let mynot = function false -> true | _ -> false;;

let iszero = function 0 -> true | _ -> false;;

let rec length = function _::tl -> \+length tl | _ -> 0;;

let rec sum = function hd::tl -> hd+sum tl | _ -> 0;;

let swap = function x::y::1 -> y::x::1 | other -> other;;

let ord_swap = function
    x::Y::tl as 1 -> if x>y then y::x::tl else 1
  |other -> other;;
\end{lstlisting}
