\chapter{Linguaggi di programmazione di alto livello}\label{ch:lpoo}
I motivi della creazione ed utilizzo di un linguaggio di programmazione di alto
livello sono di fornire una descrizione precisa, ovvero una specifica formale,
e di offrire un interpretazione tramite interprete da compilare.

Le parti principali di uno specifico linguaggio sono la sintassi e la semantica,
la quale può essere statica o dinamica.

\section{Linguaggi staticamente tipati}
Sono provvisti di semantica statica, legata alla nozione di \emph{tipo statico},
troviamo:
\begin{itemize}
  \item \textbf{statements:} devono essere usati in maniera consistente rispetto
    ai vari tipi di valori;
  \item \textbf{variabili:} devono essere dichiarate ed usate in maniera consistente
    rispetto alla loro dichiarazione.
\end{itemize}
\subsection{Vantaggi}
\begin{itemize}
  \item gli errori vengono rilevati prima, comportando meno errori dinamici;
  \item efficienza.
\end{itemize}

\section{Linguaggi di programmazione tipati}
I linguaggi di programmazione tipati non sono provvisti di semantica statica, utilizzano inconsistemente operatori,
statement e variabilir; ma generano errori dinamici.
Sono solitamente più semplici ed espressivi.

