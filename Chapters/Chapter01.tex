\chapter{Introduzione agli elementi di un linguaggio di programmazione}\label{ch:elementi}
I motivi della creazione ed utilizzo di un linguaggio di programmazione di alto
livello sono di fornire una descrizione precisa, ovvero una specifica formale,
e di offrire un interpretazione tramite interprete da compilare.

Le parti principali di uno specifico linguaggio sono la sintassi e la semantica,
la quale può essere statica o dinamica.

\section{Linguaggi staticamente tipati}
Sono provvisti di semantica statica, legata alla nozione di \emph{tipo statico},
la compilazione avviene   \emph{prima} dell'esecuzione del programma.

In un linguaggio staticamente tipato, gli operatori e gli \textit{statements}
devono essere consisenti con il tipo di valore e le variabili devono essere
dichiarate ed usate consistentemente rispetto la loro dichiarazione.

I vantaggi risiedono nella preventiva rilevazione degli errori e nell'efficienza.

\section{Linguaggi di programmazione dinamicamente tipati}
I linguaggi di programmazione dinamicamente tipati sono compilati \textbf{durante}
l'esecuzione del programma, non sono provvisti di semantica statica, utilizzano
inconsistemente operatori, statements e variabilii; ma generano errori dinamici.
Sono solitamente più semplici ed espressivi.

\subsection{Esempi di errori}
\begin{lstlisting}[language=Java, caption={Errore di sintassi}]
x = ;
\end{lstlisting}

\begin{lstlisting}[language=Java, caption={Errore statico}]
int x=0;
\end{lstlisting}

\begin{lstlisting}[language=Java, caption={Errore Dinamico}]
x = null;
if(y<0) y=1; else y=x.value;
\end{lstlisting}
