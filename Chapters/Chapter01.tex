\chapter{Introduzione agli elementi di un linguaggio di programmazione}\label{ch:elementi}
I motivi della creazione ed utilizzo di un linguaggio di programmazione di
alto livello sono:
di fornire una descrizione precisa, ovvero una specifica formale;
offrire un interpretazione tramite interprete da compilare.

Le caratteristiche principali che categorizzano i linguaggi di
programmazione sono la sintassi e la semantica, la quale può essere
statica o dinamica.

\section{Linguaggi staticamente tipati}
Nei linguaggi tipizzati staticamente il \emph{tipo} di variabile viene
stabilito nel codice sorgente, per cui si rende necessario che:

\begin{itemize}
  \item gli operatori e le assegnazioni devono essere usati coerentemente
    con il \emph{tipo} dichiarato;
  \item le variabili siano usate consistentemente rispetto la loro
    dichiarazione.
\end{itemize}

I vantaggi della staticità risiedono nella preventiva rilevazione degli
errori e nell'efficienza di calcolo risultante dalla coerenza tra codice
e compilatore.

\section{Linguaggi dinamicamente tipati}
Nei linguaggi di programmazione dinamicamente tipati le variabili sono
assegnate ai \emph{tipi} durente l'esecuzione del programma, ne consegue
che:
\begin{itemize}
  \item la semantica statica non è definita;
  \item un utilizzo incosistenze di variabili, operazioni o assegnazioni
    generano errori dinamici basati sui tipi.
\end{itemize}

I linguaggi dinamici per questi motivi risultano essere più semplici
ed espressivi.

\subsection{Esempi di errori}
\begin{lstlisting}[language=Java, caption={Errore di sintassi}]
x = ;
\end{lstlisting}
Un errore sintattico generico a molti linguaggi è un espressione
formattata erroneamente.


\begin{lstlisting}[language=Java, caption={Errore statico}]
int x=0;
if(y<0) x=3; else x="three"
\end{lstlisting}
In un linguaggio statico come \emph{Java} l'esempio precedente darebbe
un errore in quanto una stringa non può essere convertita in un tipo
intero.

\begin{lstlisting}[language=Java, caption={Errore Dinamico}]
x = null;
if(y<0) y=1; else y=x.value;
\end{lstlisting}
L'esempio, per $y>0$, darebbe un errore dinamico sia in \emph{Java}, che
in un linguaggio dinamico come \emph{Javascript}.
