\chapter{Analisi sintattica}
\begin{theorem}
  L' \textbf{analisi sintattica} è definita sull'analisi lessicale, risolve:
  \begin{itemize}
    \item il riconoscimento di una sequenza di \emph{lexems/tokens} come
      validi se rispettano alcune regole sintattiche;
    \item la costruzione, in caso di successo, di una rappresentazione
      astratta della sequenza riconosciuta per eseguire delle operazioni su
      di essa.
  \end{itemize}
\end{theorem}

\section{Parser}
\begin{theorem}
  Un \textbf{parser} è un programma che esegue l'analisi sintattica.
\end{theorem}

\subsection{Parsers per i linguaggi di programmazione}
I parser per i linguaggi di programmazione, riconoscono i \emph{token}
generati da un \emph{tokenizer}, mentre le regole sintattiche sono definite
formalmente da una \emph{grammatica}.

I parser generano:
\begin{itemize}
  \item un albero \textbf{parse/deviation}, una rappresentazione meno astratta
    della sequenza analizzata;
  \item un \textbf{albero sintattico astratto} (abstract syntax tree, AST),
    una rappresentazione pi
     astratta della sequenza analizzata.
\end{itemize}

Inoltre i parser possono essere scritti a mano, oppure generati automaticamente
da una grammatica con specifici software come \emph{ANTLR} o \emph{BISON}.

\paragraph{Primo esempio con sintassi C/Java/C++}
Token analizzati:
\begin{itemize}
  \item \emph{IDENTIFIER}: con il nome $"x2"$;
  \item \emph{ASSIGN\_OP};
  \item \emph{INT\_NUMBER}: con il valore di $34$;
  \item \emph{STATEMENT\_TERMINATOR}.
\end{itemize}

Il parser ritornerà errore dato che la sequenza non è riconosciuta oppure
uno o più messaggi di errore sono presenti.


\paragraph{Secondo esempio con sintassi C/Java/C++}
Token analizzati:
\begin{itemize}
  \item \emph{IDENTIFIER}: con il nome $"x2"$;
  \item \emph{ASSIGN\_OP};
  \item \emph{INT\_NUMBER}: con il valore di $34$;
  \item \emph{ADD\_OP};
  \item \emph{INT\_NUMBER}: con il valore di $10$;
  \item \emph{STATEMENT\_TERMINATOR}.
\end{itemize}

Il parser riconosce la sequenza e genera un \emph{AST}.

\begin{figure}[ht]
    \centering
    \incfig{ast}
    \caption{L'AST generato della sequenza precedente.}
    \label{fig:ast-2}
\end{figure}

\section{Context free (CF) grammars}
Le grammatiche \emph{context free}, sono il formalismo più diffuso per definire
le regole sintattiche di un linguaggio di programmazione.
Sono più espressive di un espressione regolare e sono basate sulla
concatenazione d unzione di più nomi e su definizioni ricorsive.

\paragraph{Un primo esempio:}
Una grammatica \textbf{CF} per espressioni semplici:
\begin{lstlisting}[language=Java, caption={CF per espressioni semplici}]
Exp ::= Num | Exp '+' Exp | Exp '*' Exp | '(' Exp ')'
Num ::= '0' | '1'
\end{lstlisting}


