\chapter{Espressioni regolari}
Le espressioni regolari sono un formalismo comunamente utilizzato per
definire linguaggi semplici.

\begin{theorem}
  La definizione induttiva di un espressione regolare su un alfabeto $A$:
  \paragraph{Base:}
  \begin{itemize}
    \item $0$ è un espressione regolare di $A$;
    \item $\lambda$ è un espressione regolare di $A$;
    \item per ogni $\sigma\in A$, $\sigma$ è un espressione regolare
      in $A$.
  \end{itemize}
  
  \paragraph{Passo induttivo:}
  \begin{itemize}
    \item se $e_1$ ed $e_2$ sono espressioni regolare di $A$,

      allora $e_1|e_2$ è un espressione regolare di $A$;
    \item se $e_1$ ed $e_2$ sono espressioni regolare di $A$,

      allora $e_1 e_2$ è un espressione regolare di $A$;
    \item se $e$ è un espressione regolare di $A$,

      allora $e^\star$ è un espressione regolarare di $A$.
  \end{itemize}
\end{theorem}
