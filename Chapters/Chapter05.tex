\chapter{Context free (CF) grammars}
Le grammatiche \emph{context free}, sono il formalismo più diffuso per definire
le regole sintattiche di un linguaggio di programmazione.
Sono più espressive di un espressione regolare e sono basate sulla
concatenazione d unzione di più nomi e su definizioni ricorsive.

\paragraph{Un primo esempio:}
\begin{lstlisting}[language=Java, caption={Una grammatica \textbf{CF} per espressioni semplici}]
Exp ::= Num | Exp '+' Exp | Exp '*' Exp | '(' Exp ')'
Num ::= '0' | '1'
\end{lstlisting}

\paragraph{Da notare che:}
\emph{Num} è definito nella grammatica solo per completezza, infatti i token
\emph{Num} sono definiti separatamente da un espressione regolare.

\begin{lstlisting}[language=Java, caption={Esempio rivisitato di una grammatica
\textbf{CF} per espressioni semplici}]
Exp ::= Num | Exp '+' Exp | Exp '*' Exp | '(' Exp ')'
NUM definito da 0|1
\end{lstlisting}

\paragraph{Notazione:}
In \emph{Exp} è maiuscolo solo il primo carattere: è definito nella grammatica.
In \emph{NUM} tutte le lettere sono maiuscole; è definito separatamente da un
espressione regolare.

\section{Terminologia delle grammatiche CF}
\begin{lstlisting}[language=Java, caption={Esempio}]
Exp ::= Num | Exp '+' Exp | Exp '*' Exp | '(' Exp ')'
Num ::= '0' | '1'
\end{lstlisting}

\subsection{Terminologia per la grammatica G}
Per $G=(T,N,P)$:
\begin{itemize}
  \item $\{'+','*','(',')','0','1'\}$ è l'insieme $T$ dei \textbf{simboli
    terminali};
  \item $\{Exp,Num\}$ sono l'insieme $N$ di simboli \textbf{non terminali};
  \item $\{(Exp,Num),(Exp,Exp,'+'Exp),(Exp,Exp '*'Exp),(Exp,'('Exp')'),(Num,'0'
    ),(Num,'1')\}$ è l'insieme $P$ di produzioni.
\end{itemize}
\paragraph{Da notare che:}
\begin{itemize}
  \item ogni simbolo non terminale corrisponde ad un linguaggio; i linguaggi
    sono definiti come unioni di concatenazioni;
  \item i simboli terminali sono \emph{lexems} dei linguaggi definiti dalla
    grammatica;
  \item le produzioni hanno forma $(B,\alpha)$, per $B\in\N\cap\alpha\in
    (T\cup N)^*$
\end{itemize}

\section{Grammatiche come definizione induttiva di linguaggi}
\subsection{Primo esempio}
\begin{lstlisting}[language=Java, caption={Esempio}]
Exp ::= Num | Exp '+' Exp | Exp '*' Exp | '(' Exp ')'
Num ::= '0' | '1'
\end{lstlisting}

\subsubsection{Definizione induttiva di linguaggi}
\[
  Exp=Num\cup(Exp\cdot\{"+"\}\cdot Exp)\cup(Exp\cdot\{"*"\}\cdot Exp)\cup(\{"("
  \}\cdot Exp\cdot\{")"\})
\]
\[
  Num=\{"0"\}\cup\{"1"\}
\]

\paragraph{Da notare che:}
\begin{itemize}
  \item $Exp=Num\cup\dots$ è il caso base per $Exp$: un numero è un
    espressione;
  \item $Exp$ è definito su di $Num$, $Num$ è definito esclusivamente per casi
    base.
\end{itemize}

\subsection{Secondo esempio}
\begin{lstlisting}[language=Java, caption={Esempio}]
Exp ::= Term | Exp '+' Term | Exp '*' Term
Term ::= '(' Exp ')' | Num
Num ::= '0' | '1'
\end{lstlisting}

\paragraph{Da notare che:}
Le definizioni di $Exp$ e $Term$ sono ricorsive reciprocamente.

\section{Derivazioni}
\begin{lstlisting}[language=Java, caption={Esempio}]
Exp ::= Num | Exp '+' Exp | Exp '*' Exp | '(' Exp ')'
Num ::= '0' | '1'
\end{lstlisting}

\subsection{Linguaggi generati da una grammatica}
\begin{itemize}
  \item Una grammatica genera un linguaggio per ogni simbolo non terminale;
  \item la grammatica precedente genera due linguaggi $L_{\text{Exp}}$ e
    $L_{\text{Num}}$;
  \item il linguaggio per $Num$ è relativamente semplice: $L_{\text{Num}}=
    \{"0","1"\}$.
\end{itemize}

\paragraph{Da notare che:}
per definire $L_{\text{Num}}$ e per dimostrare che $"1+0"\in L_{\text{Exp}}$ e
che $"1+*("\not\in L_{\text{Exp}}$ si rendono necessarie la \textbf{derivazione
a passo singolo e la derivazione a passi multipli}.

\subsection{Derivazione ad un passo}
\begin{itemize}
  \item $Exp\rightarrow Exp'*'Exp$ è usata la produzione $(Exp,Exp '*'Exp)$;
  \item $Exp'*'Exp\rightarrow Num'*'Exp$ è usata la produzione $(Exp,Num)$;
  \item $Num'*'Exp\rightarrow Num'*'Num$ è usata la produzione $(Exp,Num)$;
  \item $Num'*'Num\rightarrow '0''*'Num$ è usata la produzione $(Num,'0')$;
  \item $'0''*'Num\rightarrow '0''*''1'$ è usata la produzione $(Num,'1')$;
\end{itemize}

\paragraph{Da notare che:}
Non esiste alcuna derivazione da $'0''*''1'$ dato che nessuna produzione può
essere usata; $'0''*''1'$ è la stringa $0*1$ che appartiene a $L_{\text{Exp}}$.

\subsection{Definizioni di derivazione}
\subsubsection{Derivazione ad un passo}
\begin{theorem}
  La derivazione ad un passo per la grammatica $G=(T,N,P)$:
  \begin{itemize}
    \item possiede una forma $\alpha_1 B\alpha_2\rightarrow\alpha_1\gamma
      \alpha_2$;
    \item $\alpha_1,\alpha_2\in(T\cup N)^*$;
    \item $(B,\gamma)\in P$ ovvero $(B,\gamma)$ in produzione.
  \end{itemize}
\end{theorem}

\subsubsection{Derivazione a più passi}
\begin{theorem}
  La chiusura transitiva di $\rightarrow$:
  \begin{itemize}
    \item il caso base: se $\gamma_1\rightarrow\gamma_2$, allora $\gamma_1
      \rightarrow^+\gamma_2$;
    \item caso induttivo: se $\gamma_1\rightarrow\gamma_2$, e $\gamma_2
      \rightarrow^+\gamma_3$, allora $\gamma_1\rightarrow^+\gamma_3$.
  \end{itemize}
\end{theorem}

\subsubsection{Linguaggio generato}
Il linguaggio $L_B$ generato da $G=(T,N,P)$ per i non terminali $B\in N$:
\begin{itemize}
  \item tutte le stringhe di terminali che possono essere derivati in uno
    o più passaggi da $B$;
  \item formalmente: $L_B=\{u\mid B\rightarrow^+ u\}$.
\end{itemize}
