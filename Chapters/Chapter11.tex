\chapter{Programmazione orientata agli oggetti}
Il modello su cui si basa il paradigma è l'invio di messaggi attraverso gli
oggetti, quando si lancia un programma si invia un messaggio ad un oggetto.

\section{Oggetti}
Un oggetto è identificato da un nome, mantiene degli stati interni che sono
solitamente nascosti ed espone i cosiddetti \emph{metodi di istanza}, l'
interfaccia con cui si può interagire con l'oggetto.

L'invocazione di un metodo di istanza implica che:
\begin{itemize}
  \item gli stati interni dell'oggetto possono venire modificati;
  \item può essere necessario passare degli argomenti alle funzioni;
  \item la funzione può ritornare un valore.
\end{itemize}

\begin{lstlisting}[
  language=Java,
  escapeinside={(£}{£)},
  caption={
    Sintassi di un oggetto
  }
]
Exp ::= Exp '.' MID '(' (Exp ( ',' Exp)*)? ')'
\end{lstlisting}

Dove \emph{MID} è il nome del metodo di istanza.

I metodi di istanza possono essere di ispezione, di modifica, o entrambi.

\subsection{Stato interno}
Solitamente lo stato interno di un oggetto non è esposto, consiste di \emph{
campi}: solitamente chiamati \emph{variabili di istanza} o \emph{attributi},
i quali salvano i dati dell'oggetto, sono solitamente modificabili.
