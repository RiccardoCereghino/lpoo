\chapter{Programmazione orientata agli oggetti}
Il modello su cui si basa il paradigma è l'invio di messaggi attraverso gli
oggetti, quando si lancia un programma si invia un messaggio ad un oggetto.

\section{Oggetti}
Un oggetto è identificato da un nome, mantiene degli stati interni che sono
solitamente nascosti ed espone i cosiddetti \emph{metodi di istanza}, l'
interfaccia con cui si può interagire con l'oggetto.

L'invocazione di un metodo di istanza implica che:
\begin{itemize}
  \item gli stati interni dell'oggetto possono venire modificati;
  \item può essere necessario passare degli argomenti alle funzioni;
  \item la funzione può ritornare un valore.
\end{itemize}

\begin{lstlisting}[
  language=Java,
  escapeinside={(£}{£)},
  caption={
    Sintassi di un oggetto
  }
]
Exp ::= Exp '.' MID '(' (Exp ( ',' Exp)*)? ')'
\end{lstlisting}

Dove \emph{MID} è il nome del metodo di istanza.

I metodi di istanza possono essere di ispezione, di modifica, o entrambi.

\subsection{Stato interno}
Solitamente lo stato interno di un oggetto non è esposto, consiste di \emph{
campi}: solitamente chiamati \emph{variabili di istanza} o \emph{attributi},
i quali salvano i dati dell'oggetto, sono solitamente modificabili.

\section{Classi}
Una classe fornisce un implementazione per gli oggetti dello stesso tipo, gli
oggetti possono essere creati dinamicamente dalle classi, ogni oggetto creato
da una classe $C$ sono detti \emph{istanza} di $C$, il numero di istanze di un
oggetto a runtime può crescere o diminuire, perchè possono essere deallocate
automaticamente o manualmente.

Tutte le istanze condividono gli stessi metodi di istanza, ma ogni istanza ha
i propri stati interni.

Oggetti creati da una classe $C$ hanno \emph{tipo dinamico} $C$, in un
linguaggio staticamente tipato si dice anche \emph{tipo statico}.

\begin{lstlisting}[
  language=Java,
  escapeinside={(£}{£)},
  caption={
    Un esempio di classe in Java
  }
]
public class TimerClass {
  private int time; // variabile di istanza

  // metodo di istanza
  public boolean isRunning() {
    return this.time > 0;
  }

  public int getTime() {
    return this.time;
  }

  public void tick() {
    if (this.time > 0)
      this.time--;
  }

  public int reset (int minutes) {
    if (minutes < 0 || minutes > 60)
      throw new IllegalArgumentException();
    int prevTime = this.time;
    this.time = minutes * 60;
    return prevTime;
  }
}
\end{lstlisting}

\begin{lstlisting}[
  language=Java,
  escapeinside={(£}{£)},
  caption={
    Utilizzo della classe
  }
]
// Creo un nuovo oggetto e lo assegno a t1
TimerClass t1 = new TimerClass();
// Chiamo il metodo di istanza reset
t1.reset(1);
// Creo un nuovo oggetto e lo assegno a t2
TimerClass t2 = new TimerClass();
// Chiamo il metodo di istanza reset
t2.reset(2)
\end{lstlisting}

\subsection{Parola chiave: this}
La parola chiave \textbf{this} fa riferimento all'oggetto su cui vengono
chiamati i metodi.

\begin{lstlisting}[
  language=Java,
  escapeinside={(£}{£)},
  caption={
    Esempio di utilizzo di this
  }
]
public int reset(int minutes) {
  if (minutes < 0 || minutes > 60)
    throw new IllegalArgumentException();
  int prevTime = this.time;
  this.time = minutes * 60;
  return prevTime;
}
\end{lstlisting}

\subsection{Information Hiding}
La visibilità di metodi o campi di un oggetto può essere definita:
\begin{itemize}
  \item \textbf{private} il metodo o il campo è visibile solo all'interno della
    classe;
  \item \textbf{public} il metodo o il campo è visibile fuori dalla classe;
  \item \textbf{public class} la dichiarazione della classe è visibile in tutto
    il programma.
\end{itemize}

\subsection{Eccezioni}
La dichiarazione \textbf{throw} si utilizza quando deve essere segnalato un
errore, la sintassi è $'throw'Exp$, l'argomento di throw deve essere un
eccezione, la quale è un oggetto particolare.
\begin{lstlisting}[
  language=Java,
  escapeinside={(£}{£)},
  caption={
    Esempio di eccezione in Java
  }
]
Throwable ex;
...
throw 42; // errore
throw new NullPointerException();
throw ex;
\end{lstlisting}

\subsection{Asserzioni}
La sintassi delle asserzioni è $'assert'Exp$, dove $Exp$ deve essere un
booleano, si usano per documentare, testare e validare gli oggetti.
\begin{lstlisting}[
  language=Java,
  escapeinside={(£}{£)},
  caption={
    Esempio di asserzione in Java
  }
]
TimerClass tl = new TimerClass();
t1.reset(1);
int seconds = 0;
while (t1.isRunning()) {
  t2.tick();
  seconds++;
}
assert seconds == 60;
assert !t1.isRunning();
\end{lstlisting}

\subsection{Oggetti come valori}
Nei linguaggi di programmazione orientati agli oggetti, gli oggetti sono valori
di primo ordine, ovvero:
\begin{itemize}
  \item possono essere passati a variabili;
  \item possono essere passati come argomenti;
  \item possono essere ritornati come valori.
\end{itemize}

Gli oggetti sono rappresentati dalla loro \textbf{identità}, ovvero l'indirizzo
di memoria dello heap in cui l'oggetto è salvato.
Motivo per cui gli oggetti sono passati per referenza.
\begin{lstlisting}[
  language=Java,
  escapeinside={(£}{£)},
  caption={
    Esempio di oggetti come valori
  }
]
TimerClass t1 = new TimerClass();
TimerClass t2 = t1;
TimerClass t3 = null;
assert t1 == t2 && t1 != t3; // true
\end{lstlisting}

\subsection{Tipi e sottotipi}
Le classi definiscono anche tipi statici, chiamati in Java \emph{reference
type} ed in altri contesti \emph{object type}.

Ciò significa che se un espressione $e$ è di tipo statico \emph{TimerClass},
allora $e$ potraà tornare o un istanza di \emph{TimerClass}, o un istanza
di sottotipo di \emph{TimerClass} o \emph{null}.

\subsection{Design by contract}
Pre-condizioni, post-condizioni e invarianti su di un metodo:
\begin{itemize}
  \item \textbf{requires p}: è necessario che il predicato $p$ sia valido
    prima dell'esecuzione del metodo;
  \item \textbf{ensures p}: è necessario che il predicato $p$ sia valido dopo
    che il metodo è stato eseguito;
  \item \textbf{invariant per la classe C}: è un predicato che deve essere
    valido alla creazione di ogni istanza di $C$, e prima e dopo l'esecuzione
    di ogni metodo di $C$.
\end{itemize}

\begin{lstlisting}[
  language=Java,
  escapeinside={(£}{£)},
  caption={
    Esempio di Design By Contract
  }
]
public class TimerClass {
  private int time;
  /* invariant 0 <= time && time <= 3600; */

  public int reset(int minutes)
  /*  requires 0 <= minutes && minutes <= 60;
      ensures  result == old(this.time)
               && this.time == minutes * 60; */
  {
    if (minutes < 0 || minutes > 60)
      throw new IllegalArgumentException();
    int prevTime = this.time;
    this.time = minutes * 60;
    return prevTime;
  }

  public boolean isRunning()
  /* ensures result == this.time > 0
             && this.time == old(this.time); */
  {
    return this.time > 0;
  }
  ...
}
\end{lstlisting}

Per assicurare il funzionamento del \textbf{DBC}, si utilizza l' \emph{
information hiding} (lo stato dell'oggetto è modificabile solo tramite l'
utilizzo di metodi), la presenza di \emph{invariants} in tutti i metodi.

\section{Costruttori}
I costruttori in Java possono essere \emph{overloaded}.
\begin{lstlisting}[
  language=Java,
  escapeinside={(£}{£)},
  caption={
    Esempio di costruttori Java
  }
]
public class TimerClass (
  private int time;

  public TimerClass() {
  }
  
  public TimerClass(int minutes) {
    if (minutes < 0 || minutes > 60)
      throw new IllegalArgumentException();
    this.time = minutes * 60;
  }

  public TimerClass(TimerClass otherTimer) {
    this.time = otherTimer.time;
  }
...
}
\end{lstlisting}

\begin{lstlisting}[
  language=Java,
  escapeinside={(£}{£)},
  caption={
    Esempio di uso dei costruttori
  }
]
TimerClass t1 = new TimerClass();
TimerClass t2 = new TimerClass(42);
TimerClass t3 = new TimerClass(t2);

assert t1.getTime()==0 && t2.getTime()==42*60 &&
       t2.getTime() == t2.getTime();
\end{lstlisting}

\section{Creazione ed inizializzazione degli oggetti}
Immediatamente dopo la creazione di un oggetto dei valori di default vengono
assegnati ad ogni variabile di istanza dell'oggetto.
Il valore di default è determinato dal tipo dichiarato per quella variabile, se
presente un inizializzatore di variabile di istanza invece viene eseguito.

\begin{lstlisting}[
  language=Java,
  escapeinside={(£}{£)},
  caption={
    Esempio di iinizializzatore di variabile in Java
  }
]
public class TimerClass {
  private int time = 60;

  public TimerClass() {
  }

  public TimerClass(int minutes) {
    if (minutes < 0 || minutes > 60)
      throw new IllegalArgumentException();
    this,time = minutes * 60;
  }
  public TimerClass(TimerClass otherTimer) {
    this.time = otherTimer.time;
  }
...
}
\end{lstlisting}

\begin{lstlisting}[
  language=Java,
  escapeinside={(£}{£)},
  caption={
    Esempio di creazione ed inizializzazione di oggetti in Java
  }
]
TimerClass timer1 = new TimerClass();
TimerClass timer2 = new TimerClass(1);

assert timer1.getTime() == timer2.getTime();
\end{lstlisting}

\subsection{Costruttori}
Multipli costruttori possono essere definiti, gli stessi devono differire nel
numero o nel tipo di parametri, un costruttore di default è definito se
nessun costruttore è presente, non ha parametri ed è vuoto.

Un costruttore può essere invocato esplicitamente in un altro costruttore,
l'invocazione può essere eseguita \textbf{solo nella prima linea}.

\emph{'this' '(' (Exp ( ',' Exp)*)?')'}

Da notare che i campi degli oggetti non possono essere aggiunti o rimossi
a \emph{runtime}, i campi non opzionali devono avere sempre un valore definito
mentre i campi opzionali possono avere un valore non definito, in Java per
valore indefinito si intende \textbf{null}.
\begin{lstlisting}[
  language=Java,
  escapeinside={(£}{£)},
  caption={
    (1) Esempio di invocazione dei costruttori esplicita
  }
]
public class Person {
  private String name;
  private String address;

  public Person(String name) {
    if(name == null)
      throw new NullPointerException();
    this.name = name;
  }

  public Person (String name, String address) {
    this(name);
    this.address = address;
  }

  public String getName() {
    return name;
  }

  public String getAddress() {
    return address;
  }
}
\end{lstlisting}

\begin{lstlisting}[
  language=Java,
  escapeinside={(£}{£)},
  caption={
    (2) Esempio di invocazione dei costruttori esplicita
  }
]
Person sam = new Person("Samuele");
Person sim = new Person("Simone","Genova");
assert sam.getAddress()=null && sim.getAddress()!=null;
\end{lstlisting}

\section{Variabili di classe}
In java ci sono le variabili di istanza, che sono gli attributi degli oggetti e
le variabili di classe che sono gli attrivuti della classe.
La sintassi (\emph{CID}:class identifier, \emph{FID}:field identifier):
\begin{itemize}
  \item \textbf{field read:} \emph{CID *.*FID}
  \item \textbf{field update:} \emph{CID '.'FID '='Exp}
\end{itemize}
\begin{lstlisting}[
  language=Java,
  escapeinside={(£}{£)},
  caption={
    (1) Esempio di variabili di classe
  }
]
public class Item {
  private static long nextSN;
  private int price;
  private long SerialNumber;
  
  public Item(int price) {
    if(price < 0)
      throw new IllegalArgumentException();
    this.price = price;
    this.serialNumber = Item.nextSN++;
  }

  public int getPrice() {
    return this.price;
  }

  public long getSerialNumber() {
    return this.SerialNumber;
  }
}
\end{lstlisting}

\begin{lstlisting}[
  language=Java,
  escapeinside={(£}{£)},
  caption={
    (1) Esempio di variabili di classe
  }
]
Item item1 = new Item(61_50);
Item item2 = new Item(14_00);

assert item1.getPrice() == 61_50 && item1.getSerialNumber() == 0;
assert item2.getPrice() == 14_00 && item2.getSerialNumber() == 0;
\end{lstlisting}
