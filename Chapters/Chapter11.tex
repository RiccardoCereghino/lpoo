\chapter{Programmazione orientata agli oggetti}
Il modello su cui si basa il paradigma è l'invio di messaggi attraverso gli
oggetti, quando si lancia un programma si invia un messaggio ad un oggetto.

\section{Oggetti}
Un oggetto è identificato da un nome, mantiene degli stati interni che sono
solitamente nascosti ed espone i cosiddetti \emph{metodi di istanza}, l'
interfaccia con cui si può interagire con l'oggetto.

L'invocazione di un metodo di istanza implica che:
\begin{itemize}
  \item gli stati interni dell'oggetto possono venire modificati;
  \item può essere necessario passare degli argomenti alle funzioni;
  \item la funzione può ritornare un valore.
\end{itemize}

\begin{lstlisting}[
  language=Java,
  escapeinside={(£}{£)},
  caption={
    Sintassi di un oggetto
  }
]
Exp ::= Exp '.' MID '(' (Exp ( ',' Exp)*)? ')'
\end{lstlisting}

Dove \emph{MID} è il nome del metodo di istanza.

I metodi di istanza possono essere di ispezione, di modifica, o entrambi.

\subsection{Stato interno}
Solitamente lo stato interno di un oggetto non è esposto, consiste di \emph{
campi}: solitamente chiamati \emph{variabili di istanza} o \emph{attributi},
i quali salvano i dati dell'oggetto, sono solitamente modificabili.

\section{Classi}
Una classe fornisce un implementazione per gli oggetti dello stesso tipo, gli
oggetti possono essere creati dinamicamente dalle classi, ogni oggetto creato
da una classe $C$ sono detti \emph{istanza} di $C$, il numero di istanze di un
oggetto a runtime può crescere o diminuire, perchè possono essere deallocate
automaticamente o manualmente.

Tutte le istanze condividono gli stessi metodi di istanza, ma ogni istanza ha
i propri stati interni.

Oggetti creati da una classe $C$ hanno \emph{tipo dinamico} $C$, in un
linguaggio staticamente tipato si dice anche \emph{tipo statico}.

\begin{lstlisting}[
  language=Java,
  escapeinside={(£}{£)},
  caption={
    Un esempio di classe in Java
  }
]
public class TimerClass {
  private int time; // variabile di istanza

  // metodo di istanza
  public boolean isRunning() {
    return this.time > 0;
  }

  public int getTime() {
    return this.time;
  }

  public void tick() {
    if (this.time > 0)
      this.time--;
  }

  public int reset (int minutes) {
    if (minutes < 0 || minutes > 60)
      throw new IllegalArgumentException();
    int prevTime = this.time;
    this.time = minutes * 60;
    return prevTime;
  }
}
\end{lstlisting}

\begin{lstlisting}[
  language=Java,
  escapeinside={(£}{£)},
  caption={
    Utilizzo della classe
  }
]
// Creo un nuovo oggetto e lo assegno a t1
TimerClass t1 = new TimerClass();
// Chiamo il metodo di istanza reset
t1.reset(1);
// Creo un nuovo oggetto e lo assegno a t2
TimerClass t2 = new TimerClass();
// Chiamo il metodo di istanza reset
t2.reset(2)
\end{lstlisting}

\subsection{Parola chiave: this}
La parola chiave \textbf{this} fa riferimento all'oggetto su cui vengono
chiamati i metodi.

\begin{lstlisting}[
  language=Java,
  escapeinside={(£}{£)},
  caption={
    Esempio di utilizzo di this
  }
]
public int reset(int minutes) {
  if (minutes < 0 || minutes > 60)
    throw new IllegalArgumentException();
  int prevTime = this.time;
  this.time = minutes * 60;
  return prevTime;
}
\end{lstlisting}

\subsection{Information Hiding}
La visibilità di metodi o campi di un oggetto può essere definita:
\begin{itemize}
  \item \textbf{private} il metodo o il campo è visibile solo all'interno della
    classe;
  \item \textbf{public} il metodo o il campo è visibile fuori dalla classe;
  \item \textbf{public class} la dichiarazione della classe è visibile in tutto
    il programma.
\end{itemize}

\subsection{Eccezioni}
La dichiarazione \textbf{throw} si utilizza quando deve essere segnalato un
errore, la sintassi è $'throw'Exp$, l'argomento di throw deve essere un
eccezione, la quale è un oggetto particolare.
\begin{lstlisting}[
  language=Java,
  escapeinside={(£}{£)},
  caption={
    Esempio di eccezione in Java
  }
]
Throwable ex;
...
throw 42; // errore
throw new NullPointerException();
throw ex;
\end{lstlisting}

\subsection{Asserzioni}
La sintassi delle asserzioni è $'assert'Exp$, dove $Exp$ deve essere un
booleano, si usano per documentare, testare e validare gli oggetti.
\begin{lstlisting}[
  language=Java,
  escapeinside={(£}{£)},
  caption={
    Esempio di asserzione in Java
  }
]
TimerClass tl = new TimerClass();
t1.reseet(1);
int seconds = 0;
while (t1.isRunning()) {
  t2.tick();
  seconds++;
}
assert seconds == 60;
assert !t1.isRunning();
\end{lstlisting}

\subsection{Oggetti come valori}
Nei linguaggi di programmazione orientati agli oggetti, gli oggetti sono valori
di primo ordine, ovvero:
\begin{itemize}
  \item possono essere passati a variabili;
  \item possono essere passati come argomenti;
  \item possono essere ritornati come valori.
\end{itemize}

Gli oggetti sono rappresentati dalla loro \textbf{identità}, ovvero l'indirizzo
di memoria dello heap in cui l'oggetto è salvato.
Motivo per cui gli oggetti sono passati per referenza.
\begin{lstlisting}[
  language=Java,
  escapeinside={(£}{£)},
  caption={
    Esempio di oggetti come valori
  }
]
TimerClass t1 = new TimerClass();
TimerClass t2 = t1;
TimerClass t3 = null;
assert t1 == t2 && t1 != t3; // true
\end{lstlisting}
