\chapter{Paradigmi di programmazione}
\begin{theorem}
  I paradigmi di programmazione sono lo stile/approccio nell'utilizzo di un
  linguaggio di programmazione.
\end{theorem}

Il più delle volte un paradigma di programmazione è basato un un modello
computazionale emergente.

\paragraph{Esempi principali di paradigmi}
\begin{itemize}
  \item \textbf{Imperativi:} più vicini al modello hardware, quindi basati
    sulle nozioni di istruzione e stato, dei linguaggi di esempio sono:
    \begin{itemize}
      \item \textit{proceduali:} C;
      \item \textit{orientati agli oggetti:} Java;
    \end{itemize}
  \item \textbf{dichiarativi:} basati su un modello astratto, esempi di
    linguaggi sono:
    \begin{itemize}
      \item \textit{funzionali:} \emph{ML}, basati sulla nozione di
        \emph{definizione di funzione} e \emph{applicazione di funzione};
      \item \textit{logici:} \emph{Prolog}, basati sulla nozione di \emph{
        regola logica} e \emph{query}.
    \end{itemize}
\end{itemize}

Di solito i linguaggi di programmazione implementano più di un paradigma per
favorire la flessibilità, \emph{Java, Javascript, Python} supportano sia
i paradigmi imperativi che quelli dichiarativi.

\section{Paradigma completamente funzionale}
Caratteristiche:
\begin{itemize}
  \item \textbf{programma:} le definizioni di funzioni matematiche ed un
    espressione principale;
  \item \textbf{computazione:} l'applicazione della funzione (\emph{function
    call});
  \item \textbf{nessuna nozione di stato:} nessun assegnamento di variabili,
    più in generale nessuno \emph{statement}, solo \emph{espressioni};
  \item \textbf{variabili:} parametri di funzioni o variabili locali
    contenenti valori costanti.
\end{itemize}

Terminologia:
\begin{itemize}
  \item \textbf{higher order functions:} funzioni che possono accettare
    funzioni come argomenti e possono ritornare funzioni;
  \item \textbf{lambda expression/fuunction o anonymous functions:} funzioni
    ottenuto dall'evaluazione di un espressione.
\end{itemize}

\paragraph{Le funzioni sono valori di prima classe:}
un evaluazione di espressioni può dare come risultato un altra espressione 
(first class values).

\section{OCaml}
OCaml deriva da \textbf{ML}, è un linguaggio multi-paradigma con un approccio
puramente funzionale, è staticamente tipato con inferenza di tipo:
\begin{itemize}
  \item gli errori di tipo sono rilevati staticamente;
  \item i tipi possono essere omessi nei programmi.
\end{itemize}

\begin{lstlisting}[language=Caml, caption={Sintassi}]
Exp ::= ID | NUM | Exp Exp | 'fun' Pat+ '->' Exp | UOP Exp | Exp BOP Exp | '(' Exp ')'
Pat ::= ID // pattern semplificato
\end{lstlisting}

\paragraph{Commenti:}
\begin{itemize}
  \item \textbf{ID} rappresenta gli identificatori di variabile $[a-zA-Z\_]
    [\backslash w']*$;
  \item \textbf{NUM} rappresenta i numeri naturali:
    \[
      0[bB][01][01\_]*|0[oO][0-7][0-7\_]*|0[xX][0-9a-fA-F][0-9a-fA-F\_]*|\backslash d[\backslash d\_]*
    \];
  \item \textbf{UOP} rappresenta operatori aritmetici unari $[+-]$;
  \item \textbf{BOP} rappresenta operatori aritmetici binari $[+-*\backslash]|mod$;
  \item \textbf{Pat} rappresenta i patterns, per semplicità si possono
    associare agli identificatori.
\end{itemize}

\subsection{Funzioni ed applicazioni}
\begin{itemize}
  \item esempi di funzioni anonime:
    \begin{lstlisting}[language=Caml, caption={Esempio di funzioni anonime}]
    fun x -> x+1 (* funzione di incremento *)
    fun x y -> x+y (* funzione di addizione *)
    \end{lstlisting}
  \item applicazione di funzioni:
    \begin{lstlisting}[language=Caml, caption={Esempio di applicazione di funzioni}]
    (fun x -> x+1) 3 (* il risultato sara 4 *)
    \end{lstlisting}
\end{itemize}

Inoltre prendendo in esempio:
\begin{lstlisting}[language=Caml, caption={Esempio di applicazioni}]
  exp1 exp2
\end{lstlisting}
\begin{itemize}
  \item l'evaluazione di \textbf{exp1} si aspetta come valore di ritorno una
    funzione $f$;
  \item l'evaluazione di \textbf{exp2} si aspetta come valore di ritorno un
    argomento valido $a$;
  \item l'evaluazione di \textbf{exp1 exp2} ritorna $f(a)$ ($f$ applicato ad
    $a$).
\end{itemize}

\subsection{Regole di precedenza ed associatività}
\begin{itemize}
  \item Si usano le regole standard per le espressioni aritmetiche;
  \item l'applicazione è associativa a sinistra:
    \begin{lstlisting}[language=Caml, caption={Esempio di associatività}]
    (fun x y -> x+y) 3 4 (* equivale a ((fun x y -> x+y) 3) 4 *)
    \end{lstlisting}
  \item l'applicazione ha precedenza maggiore rispetto gli operatori binari:
    \begin{lstlisting}[language=Caml, caption={Esempio di precedenza}]
    (fun x -> x*2) 1+2 (* equivale a ((fun x -> x*2) 1) +2 *)
    1+(fun x -> x*2) 1+2 (* equivale a 1+((fun x -> x*2) 1) +2 *)
    \end{lstlisting}
  \item le funzioni anonime hanno piorità più bassa rispetto le applicazioni
    e gli operatori binari:
    \begin{lstlisting}[language=Caml, caption={Esempio di precedenza}]
    fun x -> x*2 (* equivale a fun x -> x*2) *)
    fun f a -> f a (* equivale a fun f a -> (f a) *)
    \end{lstlisting}
  \item i casi limite:
    \begin{lstlisting}[language=Caml, caption={Casi limite di precedenza}]
    f + 3 (* addizione *)
    f (+3) (* applicazione *)
    f - 3 (* sottrazione *)
    f (-3) (* applicazione *)
    + f 3 (* equivale a +(f 3) *)
    - f 3 (* equivale a -(f 3) *)
    \end{lstlisting}
\end{itemize}

\subsubsection{Una sessione REPL (Read Eval Print Loop)}
\begin{lstlisting}[language=Caml, caption={I tipi possono essere inferiti dall'interprete}]
# 42;;
- : int = 42
# fun x->x*2;;
- : int -> int = <fun>
# (fun x->x+1) 2;;
- : int = 3
\end{lstlisting}

\paragraph{Sintassi}
\begin{lstlisting}[language=Caml, caption={Grammatica BNF}]
Type ::= 'int' | Type '->' Type
\end{lstlisting}

\subsubsection{Terminologia}
\begin{itemize}
  \item \textbf{int} è il tipo primitivo degli interi;
  \item \textbf{int} -> \textbf{int} è un tipo composito;
  \item \textbf{->} è un tipo \textit{costruttore}, usato per costruire tipi
    compositi da tipi più semplici;
  \item i tipi costruiti con il costruttore freccia (->) sono chiamati \emph{
    arrow types} o \emph{tipi di funzione}.
\end{itemize}

\paragraph{Significato del tipo freccia}
$t_1 -> t_2$ identifica il tipo di funzioni da $t_1$ a $t_2$ che:
\begin{itemize}
  \item piò essere applicazo ad un singolo argomento di tipo $t_1$;
  \item ritorna sempre un valore di tipo $t_2$.
\end{itemize}

\subparagraph{Osservazioni:}
\begin{itemize}
  \item il costruttore di tipo freccia è associativo a destra.
    \begin{lstlisting}[language=Caml, caption={Associatività a destra dell'operatore freccia}]
    int->int->int = int->(int->int)
    \end{lstlisting}
  \item un tipo costruttore costruisce sempre un tipo diverso rispetto a quello
    dei propri componenti:
    \[
      t_1 \text{->} t_2 \neq t_1 \qquad t_1 \text{->} t_2 \neq t_2
    \]
  \item due tipi freccia sono uguali se sono costruiti dallo stesso tipo di
    componenti:
    \[
      t_1 \text{->} t_2 = t_3 \text{->} t_4 \text{ se e solo se } t_1 = t_3 \cap t_2 = t_4
    \]
  \item dai punti prima deduciamo:
    \begin{lstlisting}[language=Caml, escapeinside={(*}{*)}]
    int->int->int (*$\neq$*) int->(int->int)
    \end{lstlisting}
\end{itemize}

\subsection{Funzioni di alto ordine}
\[
  \textbf{fun } pat_1 pat_2 \dots pat_n \text{->} exp
\]
è un abbreviazione di:
\[
  \textbf{fun } pat_1 \text{->}\textbf{fun }pat_2 \text{->}\dots \textbf{fun }pat_n \text{->} exp
\]
\subsection{Tuple}

