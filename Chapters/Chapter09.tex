\chapter{Paradigmi di programmazione}
\begin{theorem}
  I paradigmi di programmazione sono lo stile/approccio nell'utilizzo di un
  linguaggio di programmazione.
\end{theorem}

Il più delle volte un paradigma di programmazione è basato un un modello
computazionale emergente.

\paragraph{Esempi principali di paradigmi}
\begin{itemize}
  \item \textbf{Imperativi:} più vicini al modello hardware, quindi basati
    sulle nozioni di istruzione e stato, dei linguaggi di esempio sono:
    \begin{itemize}
      \item \textit{proceduali:} C;
      \item \textit{orientati agli oggetti:} Java;
    \end{itemize}
  \item \textbf{dichiarativi:} basati su un modello astratto, esempi di
    linguaggi sono:
    \begin{itemize}
      \item \textit{funzionali:} \emph{ML}, basati sulla nozione di
        \emph{definizione di funzione} e \emph{applicazione di funzione};
      \item \textit{logici:} \emph{Prolog}, basati sulla nozione di \emph{
        regola logica} e \emph{query}.
    \end{itemize}
\end{itemize}

Di solito i linguaggi di programmazione implementano più di un paradigma per
favorire la flessibilità, \emph{Java, Javascript, Python} supportano sia
i paradigmi imperativi che quelli dichiarativi.

\section{Paradigma completamente funzionale}
Caratteristiche:
\begin{itemize}
  \item \textbf{programma:} le definizioni di funzioni matematiche ed un
    espressione principale;
  \item \textbf{computazione:} l'applicazione della funzione (\emph{function
    call});
  \item \textbf{nessuna nozione di stato:} nessun assegnamento di variabili,
    più in generale nessuno \emph{statement}, solo \emph{espressioni};
  \item \textbf{variabili:} parametri di funzioni o variabili locali
    contenenti valori costanti.
\end{itemize}

Terminologia:
\begin{itemize}
  \item \textbf{higher order functions:} funzioni che possono accettare
    funzioni come argomenti e possono ritornare funzioni;
  \item \textbf{lambda expression/fuunction o anonymous functions:} funzioni
    ottenuto dall'evaluazione di un espressione.
\end{itemize}

\paragraph{Le funzioni sono valori di prima classe:}
un evaluazione di espressioni può dare come risultato un altra espressione 
(first class values).
