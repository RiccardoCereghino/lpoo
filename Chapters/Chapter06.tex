\chapter{Alberi di derivazione}
\section{Albero di derivazione (parse tree)}
\paragraph{Osservazione 1}
Le grammatiche CF sono utilizzate per definire linguaggi ed implementare
parsers, i parsers dovrebbero generare gli alberi, ma le derivazioni non
sono alberi.

\paragraph{Osservazione 2}
Un passo di derivazione è determinato da:
\begin{itemize}
  \item la produzione usata;
  \item lo specifico simbolo non termitale che rimpiazza.
\end{itemize}
Quest'ultimo punto non influenza la stringa finale dei terminali ottenuti
dalla derivazione.

\paragraph{Intuizione}
Un albero di derivazione è una generalizzazione di una derivazione a più
passaggi in modo che la stringa derivata contenga solo terminali e che i non
terminali siano rimpiazzati in parallelo.

\subsection{Esempi di alberi di derivazione (ANTLR)}
\begin{lstlisting}[language=Java, caption={Grammatica ANTLR}]
grammar SimpleExp;

Exp ::= Num | Exp '+' Exp | Exp '*' Exp | '(' Exp ')'
Num ::= '0' | '1'
\end{lstlisting}

\subsubsection{Albero di derivazione per "1*1+1"}
\begin{figure}[h]
    \centering
    \incfig{dtree-1}
    \caption{Albero di derivazione per "1*1+1""}
    \label{fig:dtree-1}
\end{figure}

\paragraph{Esercizio:}
Mostrare che $"1*1+1"\in L_{\text{Exp}}$ usando la nozione di
derivazione ad uno o più passi.
\[
  Exp\rightarrow Exp'+'Exp\rightarrow^+ Exp\times Exp'+'Num'
  \rightarrow^+ Num'*'Num'+'1'
  \rightarrow^+ \text{'1' '*' '1' '+' '1'}
\]
\paragraph{Esercizio:}
Mostrare che $"1+(\not\in L_{\text{Exp}}$:
\[
  Exp\rightarrow Exp '*' Exp\rightarrow^+ Num '*' Exp '+' Exp \rightarrow^+
\text{'1' '*' Num '+' Num}\rightarrow^+\text{'1' '*' '1' '+' '1'}
\]

\subsection{Definizione di albero di derivazione in G=(T,N,P)}
Albero di derivazione per $u\in T^*$ partendo da $B\in N$.
\begin{itemize}
  \item se un nodo è etichettato da $C$ ed ha $n$ figli $I_1,\dots,I_n$,
    allora $(C,I_1,\dots,I_n)\in P$ (ovvero, $(C,I_1,\dots,I_n)$ è una
    produzione di $G$;
  \item la radice è etichettata da $B$;
  \item $u$ è ottenuto dalla concatenaziona da sinistra a destra dii tutte le
    etichette terminali (nodi foglia).
\end{itemize}

\subsubsection{Definizione equivalente di un linguaggio generato}
Il linguaggio $L_B$ generato da $G=(T,N,P)$ per un non terminale $B\in N$ è
composto da tutte le stringhe $u$ di terminali così che esiste un albero di
derivazione per $u$ partendo da $B$.
