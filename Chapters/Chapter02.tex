\chapter{Sintassi}
\begin{theorem}
  Un alfabeto è un insieme finito non vuoto di simboli.
\end{theorem}
\begin{theorem}
  Sia una stringa in un alfabeto $A$ la successione di simboli in $u$:
  \[u:[1\dots n] \rightarrow A\]
  Sia:
  \begin{itemize}
    \item $[1\dots n] = m$, l'intervallo dei numeri naturali tale che: $1\leq m
      \geq n$;
    \item $u$ è una funzione totale;
    \item $n$ sia la lunghezza di $u: \text{length}(u)=n$.
  \end{itemize}
\end{theorem}
\begin{theorem}
\begin{theorem}
  Un programma è una stringa in un alfabeto $A$.
\end{theorem}

\section{Stringhe}
\subsection{Stringa vuota}
\[u: [1\dots0] \rightarrow A\]
Esiste un unica funzione $u: 0\rightarrow A$

Le notazioni standard di una stringa vuoto sono: $\varepsilon, \lambda$

\subsection{Stringa non vuota}
Si consideri $A=\{'a',\dots,'z'\}\cup\{'A',\dots,'Z'\}$, l'alfabeto inglese
di lettere minuscole e maiuscole.
La funzione $u:[1\dots4]\rightarrow A$ rappresenta la stringa $"Word"$ con:
\begin{itemize}
  \item $u(1)$ = 'W'
  \item $u(1)$ = 'o'
  \item $u(1)$ = 'r'
  \item $u(1)$ = 'd'
\end{itemize}

\subsection{Concatenazione di stringhe}
\begin{theorem}
  \[\text{length}(u\cdot v) = \text{length}(u) + \text{length}(v)\]
  
  Per ogni $i\in[1\dots \text{length}(u) + \text{length}(v)]$
  \[ (u\cdot v)(i) = \textbf{if } i\leq<\text{length}(u) \textbf{then } u(i)
  \textbf{else } v(i-\text{length}(u))\]
\end{theorem}
\subsubsection{Monoide}
La concatenazione è associativa, ma non commutativa.

La stringa vuota è l'identità dell'elemento.
\subsubsection{Iterazione}
La definizione di $u^n$ per induzione su $n \in \N$:
\[u^0 = \lambda \]
\[u^{n+1} = u \cdot u^n\]
Per cui $u^n$ si concatena con se stesso $n$ volte.
\end{theorem}
