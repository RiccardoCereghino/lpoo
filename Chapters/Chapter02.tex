\chapter{Stringhe}
\begin{theorem}
  Un alfabeto $A$ è un insieme finito non vuoto di simboli.
\end{theorem}
\begin{theorem}
  Sia una stringa in un alfabeto $A$ la successione di simboli in $u$:
  \[u:[1\dots n] \rightarrow A\]
  Sia:
  \begin{itemize}
    \item $[1\dots n] = m$, l'intervallo dei numeri naturali tale che:
      \[
        1\leq m\geq n;
      \]
    \item $u$ sia una funzione totale;
    \item $n$ sia la lunghezza di $u: \text{length}(u)=n$.
  \end{itemize}
\end{theorem}
\begin{theorem}
  Un programma è una stringa in un alfabeto $A$.
\end{theorem}

\section{Esempio di stringhe}
\subsection{Stringa vuota}
\[u: [1\dots0] \rightarrow A\]
Esiste un unica funzione $u: 0\rightarrow A$

Le notazioni standard di una stringa vuoto sono: $\varepsilon, \lambda$

\subsection{Stringa non vuota}
Si consideri $A=\{'a',\dots,'z'\}\cup\{'A',\dots,'Z'\}$, l'alfabeto inglese
di lettere minuscole e maiuscole.
La funzione $u:[1\dots4]\rightarrow A$ rappresenta la stringa $"Word"$ con:
\begin{itemize}
  \item $u(1)$ = 'W'
  \item $u(1)$ = 'o'
  \item $u(1)$ = 'r'
  \item $u(1)$ = 'd'
\end{itemize}

\subsection{Concatenazione di stringhe}
\begin{theorem}
  \[\text{length}(u\cdot v) = \text{length}(u) + \text{length}(v)\]
  
  Per ogni $i\in[1\dots \text{length}(u) + \text{length}(v)]$
  \[ (u\cdot v)(i) = \textbf{if } i\leq<\text{length}(u) \textbf{then } u(i)
  \textbf{else } v(i-\text{length}(u))\]
\end{theorem}
\subsubsection{Monoide}
La concatenazione è associativa, ma non commutativa.

La stringa vuota è l'identità dell'elemento.
\subsubsection{Induzione}
La definizione di $u^n$ per induzione su $n \in \N$:

Base: $u^0 = \lambda$

Passo induttivo: $u^{n+1} = u \cdot u^n$
Per cui $u^n$ si concatena con se stesso $n$ volte.
\subsection{Insiemi di stringhe}
\begin{theorem}
  Sia $A$ un alfabeto:
  \begin{itemize}
    \item $A^n =$ l'insieme di tutte le stringhe in $A$ con lunghezza $n$;
    \item $A^+ =$ l'insieme di tutte le stringhe in $A$ con lunghezza maggiore
      di $0$;
    \item $A^\star =$ l'insieme di tutte le stringhe in $A$;
    \item $A^+ = \bigcup_{n>0}A^n$;
    \item $A^\star = \bigcup_{n\geq0}A^n=A^0\cup A^+$
  \end{itemize}
\end{theorem}

\section{Linguaggio formale}
\begin{theorem}[Nozione sintattica di linguaggio]
  Un linguaggio $L$ in un alfabeto $A$ è un sottoinsieme di $A^\star$
\end{theorem}

\paragraph{Esempio:}
L'insieme $L_{\text{id}}$ di tutti gli identificatori di variabile:
\[A=\{'a',\dots,'z'\}\cup\{'A',\dots,'Z'\}\cup\{'0',\dots,'9'\}\]
\[L_{\text{id}}=\{'a','b',\dots,'a0','a1',\dots\}\]

\subsection{Composizione di operatori tra linguaggi}
Le operazioni possono essere di concatenazione o di unione:
\begin{itemize}
  \item \textbf{Concatenazione:} $L_1\cdot L_2 = \{u\cdot w|u\in L_1, w\in L_2\}$;
  \item \textbf{Unione:} $L_1\cup L_2$.
\end{itemize}

\subsection{Intuizione}
\subsubsection{Unione}
$L=L_1\cup L_2$: qualsiasi stringa $L$ è una stringa di $L_1$ o di $L_2$.
\paragraph{Esempio:}
\[L^\prime = \{'a',\dots,'z'\}\cup\{'A',\dots,'Z'\} \]

\subsubsection{Concatenazione}
$L=L_1\cdot L_2$: qualsiasi stringa $L$ è una stringa di $L_1$, seguita da una
stringa di $L_2$.
\paragraph{Esempio:}
\[\{'a','ab'\}\cdot\{\lambda,'1'\}=\{'a','ab','a1','ab1'\}\]
\[L_{\text{id}}=L^\prime\cdot A^\star \text{ con }A=\{'a',\dots,'z'\}\cup\{'A',\dots,'Z'\}\cup\{'0',\dots,'9'\} \]

\subsection{Monoide}
La concatenazione è associativa, ma non commutativa.

$A^0(=\{\lambda\})$ è l'identità dell'elemento; quindi $A^0$ \emph{non è
l'elemento neutro}, l'elemento neuro è $0=\{\}$.

\subsection{Passo induttivo}
$L^n$ è definito per induzione su $n\in\N$:
Base: $L^0=A^0(=\{\lambda\}$,

Passo induttivo: $L^{n+1}=L\cdot L^n$.

\subsection{Operatori + e *}
\begin{itemize}
  \item \textbf{Addizione:} $L^+=\bigcup_{n>0}L^n$;
  \item \textbf{Moltiplicazione:} $\star$ viene chiamata \textit{Kleen star,
    stella di Kleen}.
    \[L^\star = \bigcup_{n\geq0}L^n\]
\end{itemize}
Sono equivalenti $L^\star=L^0\cup L^+$, $L\cdot L^\star$.

\subsubsection{Intuizione}
\begin{itemize}
  \item Qualsiasi stringa di $L^+$ è ottenuta concatenando una o più stringhe
    di $L$;
  \item Qualsiasi stringa di $L^\star$ è ottenuta concatenando $0$ o più
    stringhe di $L$: \emph{Concatenando zero stringhe si ottiene la stringa
    vuota}.
\end{itemize}

